\section{Introduction}
 
The animal nervous system enables an extraordinary range of natural behaviors, and has inspired much of modern artificial intelligence.
Neural connectivities - axon-dendrite connections from one region to another - form the architecture underlying this capability.
These connectivities vary by neuron type, as well as axonic source and dendritic target structure.
Thus, characterization of the relationship between neuron type and source and target structure is an important for understanding the overall nervous system.

Viral tracing experiments - in which a viral vector expressing GFP is transduced into neural cells through stereotaxic injection - are a useful tool for understanding these connections on the mesoscale \citep{Chamberlin1998-hi,Harris2012-fw, Daigle2018-gd}.
The GFP protein moves from axon to dendrite through the process of anterograde projection, so neurons 'downstream' of the injection site will also fluoresce.
Two-photon tomography imaging can then determine the location and strength of the fluorescent signals in two-dimensional slices.
These locations can then be mapped back into three-dimensional space, and the signal is partitioned into the transduced source and merely transfected target regions \skcomment{Check}.

Several statistical models for the conversion of such experiment-specific signals into estimates of connectivity strength have been proposed \citep{Oh2014-kh, Harris2016-fn, Knox2019-ot}.
Of these, \citet{Oh2014-kh} and \citet{Knox2019-ot} model \textbf{structural connectivities} between structures.
Intuitively, these models provide some improvement over simply averaging the projection signals of injections in a given region.
However, these works model connectivities observed in wild-type mice transduced with constitutive promoters, and so are poorly suited for extension to recently developed tracing experiments that induce cell-type specific fluorescence \citep{Harris2019-mr}.
In particular, GFP promotion is induced by Cre-recombinase expression in cell-types specified by transgenic strain.
%The relation of cell-types to neuron classes is complex
Thus, this paper introduces a \textbf{cell class}-specific statistical model to deal with the diverse set of \textbf{cre-lines} described in \citet{Harris2019-mr}.

Our model is a to-our-knowledge novel estimator that takes into account both the spatial position of the labelled source, as well as the categorical cell class.
Like the previously state-of-the-art model in \citet{Knox2019-ot}, this model predicts structural connectivity as an average over positions within the structure, with nearby experiments given more weight.
However, our model weighs class-specific behavior in a particular structure against spatial position, so a nearby experiment targeting a similar cell class would be relatively upweighted, while a nearby experiment targeting a dissimilar class would be downweighted.
This model outperforms the model of  \citet{Knox2019-ot} based off of their ability to predict held-out experiments in leave-one-out cross-validation.
We then establish a lower-limit of detection, and use the trained model to estimate overall connectivity matrices for assayed each cell class.

The resulting cell-type specific connectivity matrices form a multi-way \textbf{structural connection tensor} of information about neural structure.
We do not attempt an exhaustive analysis of this data, but do manually verify several cell-type specific connectivity patterns found elsewhere in the literature, and show that these cell-type specific signals are behaving in expected ways.
Finally, we decompose the wild-type connectivity matrix into factors representing archetypal connective patterns using non-negative matrix factorization.
These components are themselves novel and of some independent interest.

Section \ref{sec:methods} gives information on the data and statistical methodology, and Section \ref{sec:results} presents our results.
These include connectivities, assessments of model fit, and subsequent analyses.
Additional information on our dataset, methods, and results are given in Supplemental Sections \ref{supp_sec:info}, \ref{supp_sec:methods}, and \ref{supp_sec:exp}, respectively.
%This model outperforms the model of \citet{Knox2019-ot}, even for wild-type experiments.
