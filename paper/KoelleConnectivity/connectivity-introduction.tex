\section{Introduction}
 
The animal nervous system enables an extraordinary range of natural behaviors, and has inspired much of modern artifical intelligence.
Neural connectivities - axon-dendrite connections from one region to another - form the architecture underlying this capability.
These connectivities vary by neuron type, as well as axonic source and dendritic target structure.
Thus, characterization of the relationship between neuron type and source and target structure is an important step to understanding the nervous system.

Viral tracing experiments - in which a viral vector expressing GFP is transduced into neural cells through stereotaxic injection - are a useful tool for understanding these connections on the mesoscale \citep{Chamberlin1998-hi,Harris2012-fw, Daigle2018-gd}.
The GFP protein moves from axon to dendrite through the process of anterograde projection, so neurons 'downstream' of the injection site will also fluoresce.
Two-photon tomography imaging can then determine the location and strength of the fluorescent signals in two-dimensional slices.
These locations can then be mapped back into three-dimensional space, and
the signal is partitioned into the transduced source and merely transfected target regions.


The conversion of such experiment-specific signals into an overall estimate of the connectivity strength of two regions is accomplished by a statistical model.
\citet{Oh2014-kh} and \citet{Knox2019-ot} describe two such methods.
Intuitively, both of these models provide some improvement over simply averaging the projection signals of injections in a given region.
\citet{} is another.
These models are evaluated based off of their ability to predict held-out experiments in leave-one-out cross validation.
A model that performs well in such validation experiments is then assumed to generate the most accurate connectivity. 

Both \citet{Oh2014-kh} and \citet{Knox2019-ot} develop models for mostly wild-type mice using a standardized vector over all experiments.
However, recent work \citep{Harris2019-mr} has extended these datasets to include viral tracing experiments inducing cell-type specific fluorescence.
This is accomplished by injecting vectors with Cre-recombinase triggered GFP promoters into transgenic mice with cell-type specific Cre-recombinase expression
Thus, the this paper extends the methodology of \citet{Oh2014-kh} and \citet{Knox2019-ot} to deal with the diverse set of cre-lines described in \citet{Harris2019-mr}.

This extension relies on a to our knowledge novel estimator that takes into account both the spatial position of the labelled source, as well as the categorical cre-label.
This model outperforms the model of \citet{Knox2019-ot}, even for wild-type experiments.

The resulting cell-type specific connectivity matrices form a multi-way \textit{neural connection tensor} of information about neural structure.
We do not attempt an exhaustive analysis of this data, but do demonstrate several basic phenomena.
First, we verify several cell-type specific patterns found elsewhere in the literature.
Second, we discover cell-type specific signals in the neural connection tensor.
Finally, we decompose the overall (wild-type) connectivity matrix into factors representing archetypal connective patterns.



\newpage
