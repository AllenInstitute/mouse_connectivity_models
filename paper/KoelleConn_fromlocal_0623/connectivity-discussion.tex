\section{Discussion}

The model presented here is a milestone in characterization of connectomes. It is the first cell-type specific whole brain projectomme for a mammalian species. It can open the door for a large umber of models linking brain structure to computational architectures. 

The method presented here is also novel. Beyond using a Nadaraya-Watson kernel regression defined in physical space, we define a cell-type space based on similarities of projections, and expand the method to work in this space as well. 

We see several opportunities for improving on our model.
Our particular task of transforming the injection and projection signal depending on cell-type is a non-linear transformation problem with categorical covariate.
Model averaging based off of cross-validation has been implemented in \citet{Gao2016-qe}, but we note that our approach makes use of a non-parametric estimator, rather than an optimization method for selecting the weights \citep{Saul2003-th}, and is applied specifically to a target-encoded feature space.
The properties of this estimator, as well as its relation to estimators fit using an optimization algorithm, are a possible future avenue of research.
Therefore, a deep model such as \citet{Lotfollahi2019-tr} could be appropriate, provided enough data was available.
With respect to the model, a Wasserstein-based measure of injection similarity per structure would combine both the physical simplicity of the centroid model while also incorporating structural knowledge.
Residual models of the above could also be considered.

The factorization of the connectivity matrix could be similarly improved.
Flattening $\mathcal C$ prior to unsupervised analysis is not necessarily recommended, but provides an easy solution for this problem.

%The major-structure specific Nadaraya-Watson estimator from $\citet{Knox2019-ot}$ nevertheless appears poorly adapted for our circumstances since it does not take into account the virus $v$. It is not straightforward to include the viral strain in either $f_{NW}$ or $f_{NNLS}$. For $f_{NW}$, it is not clear how a one-hot encoded class-membership feature should be weighted. For $f_{NNLS}$, our sample size seems too low to utilize a fixed or mixed effect, particularly since the impact of the virus depend on the particular injection region, motivating use of an interaction term.


