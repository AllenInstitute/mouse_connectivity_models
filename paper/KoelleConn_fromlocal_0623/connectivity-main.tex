\documentclass[NETN,manuscript]{stjour-new}
\setcounter{secnumdepth}{1}


%\usepackage{hyperref}
%\usepackage{microtype}
\usepackage{graphicx}
\usepackage{hyperref}
\usepackage{comment}
\usepackage{amsmath}
\usepackage{amsfonts}
\usepackage{amsthm}

\usepackage{graphics}
\usepackage{graphicx}

\usepackage{algorithm}
%\usepackage[ruled,vlined]{algorithm2e}

\usepackage{booktabs}
\usepackage{wrapfig,lipsum,booktabs}
\usepackage{pbox}
\usepackage{afterpage}
\usepackage{makecell}
\usepackage{fourier} 
\usepackage{array}
\usepackage{xr}
\usepackage{cleveref}
\usepackage{tikz-cd}

\usepackage[english]{babel}
\usepackage[utf8]{inputenc}


\usepackage{subfig}
\usepackage[export]{adjustbox}
%\usepackage{subfigure}
\usepackage{booktabs} % for professional tables
\usepackage{tikz-cd}

%\usepackage{algorithm}% http://ctan.org/pkg/algorithms
\usepackage{algpseudocode}



%\newcommand{\comment}[1]{}
\newcommand{\preprocess}{\textsc{preprocess}}
\newcommand{\Cov}{\mathrm{Cov}}
\newtheorem{proposition}{Proposition}
\newtheorem{definition}{Definition}
\newcommand{\dataset}{\mathcal{Dataset}}
\newcommand{\E}{\mathbb{E}}
\newcommand{\nnls}{\textsc{nnls}}
\newcommand{\nw}{\textsc{nw}}
\newcommand{\el}{\textsc{el}}
\newcommand{\nmf}{\textsc{nmf}}
\newcommand{\pre}{\textsc{preprocess}}
\newcommand{\mmp}[1]{\textcolor{red}{\em MMP: #1}}
\newcommand{\skcomment}[1]{({\color{blue}{SK's comment:}}\textbf{\color{blue}{#1}})}
\usepackage{hyperref}


%\setcounter{secnumdepth}{1}
% hyperref makes hyperlinks in the resulting PDF.
% If your build breaks (sometimes temporarily if a hyperlink spans a page)
% please comment out the following usepackage line and replace
% \usepackage{icml2019} with \usepackage[nohyperref]{icml2019} above.


%\tikzset{%
%  symbol/.style={
%    draw=none,
%    every to/.append style={
%      edge node={node [sloped, allow upside down, auto=false]{$#1$}}
%    },
%  },
%}

% Attempt to make hyperref and algorithmic work together better:
%\newcommand{\theHalgorithm}{\arabic{algorithm}}




\newlength\Colsep
\setlength\Colsep{10pt}
%\usetikzlibrary{arrows.meta,
%                positioning
%                }
%\externaldocument[supp-]{Supplement.tex}


\newcommand{\smcomment}[1]{({\color{red}{Stefan's comment:}}\textbf{\color{red}{#1}})}
%\newcommand{\skcomment}[1]{({\color{blue}{SK's comment:}}\textbf{\color{blue}{#1}})}
\renewcommand\theadalign{bc}
\renewcommand\theadfont{\bfseries}
\renewcommand\theadgape{\Gape[4pt]}
\renewcommand\cellgape{\Gape[4pt]}

\makeatletter
\newcommand*{\addFileDependency}[1]{% argument=file name and extension
  \typeout{(#1)}% latexmk will find this if $recorder=0 (however, in that case, it will ignore #1 if it is a .aux or .pdf file etc and it exists! if it doesn't exist, it will appear in the list of dependents regardless)
  \@addtofilelist{#1}% if you want it to appear in \listfiles, not really necessary and latexmk doesn't use this
  \IfFileExists{#1}{}{\typeout{No file #1.}}% latexmk will find this message if #1 doesn't exist (yet)
}
\makeatother

\newcommand*{\myexternaldocument}[1]{%
    \externaldocument{#1}%
    \addFileDependency{#1.tex}%
    \addFileDependency{#1.aux}%
}
%\newcommand{\comment}[1]{}
%\myexternaldocument{Supplement}
\listfiles

\articletype{Research}

%\def\taupav{\tau_{\mathrm{Pav}}}

\begin{document}


\title{Modelling the cell-type specific mesoscale murine connectome with anterograde tracing experiments}

\author[Koelle et al]% shortened version for running head, optional
{Samson Koelle \affil{1,2}, Jennifer Whitesell \affil{1}, Karla Hirokawa \affil{1},  Hongkui Zeng\affil{1}, Marina Meila\affil{2}, Julie Harris\affil{1}, Stefan Mihalas\affil{1}}

\affiliation{1}{Allen Institute for Brain Science, Seattle, WA, USA}

\affiliation{2}{Department of Statistics, University of Washington, Seattle, WA, USA}

\correspondingauthor{Stefan Mihalas}{stefanm@alleninstitute.org}

\keywords{[Connectivity, Cell-type, Mouse]}

\begin{abstract}
The Allen Brain Connectivity Atlas consists of anterograde tracing experiments targeting diverse
structures and classes of projecting neurons. Beyond anterograde tracing done in C57BL6 wildtype mice, a large fraction of these experiments are performed using transgenic cre lines. This allows access to cell-class specific connectivity information, with class defined by the transgenic lines.
However, even though the number of experiments is large, it does not come close to covering all existing cell classes in every area where they exist.
We study how much we can fill in these gaps and construct a voxel-based connectivity given the observations that nearby voxels have similar connections when they are in the same area, that connections can change dramatically at area boundaries, and that particular cell classes can have similar connections, but that this similarity is region-dependent.  

This paper describes the conversion of these experiments into class-specific connectivity matrices representing the connection between source and target structures. We introduce and validate a novel statistical model for creation of connectivity matrices. We expand a Nadaraya-Watson kernel learning method which we previous used to fill in spatial gaps, to also fill in a gaps in cell class connectivity information. To do this, we construct a "cell-class space" and combine smoothing in 3D spatial as well as in this abstract space to share information between similar neuron classes.
Using this method we construct a set of connectivity matrices using multiple levels of resolution at which discontinuities in connectivity are assumed. We show that the connectivities obtained from this model display expected cell-type and structure specific connectivities. 
Inspired by how this complexity arises from a much smaller set of genetic information during development, we also show that the wild-type connectivity matrix can be factored using a small set of factors, and we uncover the underlying latent structure.

\end{abstract}

\begin{authorsummary}
Large scale studies have described the connections between areas multiple mammalian models in ever expanding detail. Standard connectivity studies focus on the connection strength between areas. However, when describing functions at a local circuit level, there is an increasing focus on cell types.  We have recently described the importance of connection types in the cortico-thalamic system, which allows an unsupervised discovery of its hierarchical organization. In this study we focus on adding a dimension of connection type for a brain-wide mesoscopic connectivity model. Even with our massive dataset, the data in the type direction for the connectivity is quite sparse, and we had to develop methods to more reliably extrapolate in such directions, and to estimate when such extrapolations are impossible. This allows us to fill in such a connection type specific inter-areal connectivity matrix to the extent our data allows us to. 
While analysing this complex connectivity, we observed that it can be described via a small set of factors. 
While not complete, this connectivity matrix represents a large leap forward in mouse connectivity models. 
\end{authorsummary}

\newpage

\section{Introduction}
 
The animal nervous system enables an extraordinary range of natural behaviors, and has inspired much of modern artifical intelligence.
Neural connectivities - axon-dendrite connections from one region to another - form the architecture underlying this capability.
These connectivities vary by neuron type, as well as axonic source and dendritic target structure.
Thus, characterization of the relationship between neuron type and source and target structure is an important step to understanding the nervous system.

Viral tracing experiments - in which a viral vector expressing GFP is transduced into neural cells through stereotaxic injection - are a useful tool for understanding these connections on the mesoscale \citep{Chamberlin1998-hi,Harris2012-fw, Daigle2018-gd}.
The GFP protein moves from axon to dendrite through the process of anterograde projection, so neurons 'downstream' of the injection site will also fluoresce.
Two-photon tomography imaging can then determine the location and strength of the fluorescent signals in two-dimensional slices.
These locations can then be mapped back into three-dimensional space, and
the signal is partitioned into the transduced source and merely transfected target regions.


The conversion of such experiment-specific signals into an overall estimate of the connectivity strength of two regions is accomplished by a statistical model.
\citet{Oh2014-kh} and \citet{Knox2019-ot} describe two such methods.
Intuitively, both of these models provide some improvement over simply averaging the projection signals of injections in a given region.
\citet{} is another.
These models are evaluated based off of their ability to predict held-out experiments in leave-one-out cross validation.
A model that performs well in such validation experiments is then assumed to generate the most accurate connectivity. 

Both \citet{Oh2014-kh} and \citet{Knox2019-ot} develop models for mostly wild-type mice using a standardized vector over all experiments.
However, recent work \citep{Harris2019-mr} has extended these datasets to include viral tracing experiments inducing cell-type specific fluorescence.
This is accomplished by injecting vectors with Cre-recombinase triggered GFP promoters into transgenic mice with cell-type specific Cre-recombinase expression
Thus, the this paper extends the methodology of \citet{Oh2014-kh} and \citet{Knox2019-ot} to deal with the diverse set of cre-lines described in \citet{Harris2019-mr}.

This extension relies on a to our knowledge novel estimator that takes into account both the spatial position of the labelled source, as well as the categorical cre-label.
This model outperforms the model of \citet{Knox2019-ot}, even for wild-type experiments.

The resulting cell-type specific connectivity matrices form a multi-way \textit{neural connection tensor} of information about neural structure.
We do not attempt an exhaustive analysis of this data, but do demonstrate several basic phenomena.
First, we verify several cell-type specific patterns found elsewhere in the literature.
Second, we discover cell-type specific signals in the neural connection tensor.
Finally, we decompose the overall (wild-type) connectivity matrix into factors representing archetypal connective patterns.



\newpage

\newpage
\section{Methods}
\label{sec:methods}

We estimate and analyze cell class-specific connectivity functions using models trained on murine brain viral tracing experiments.
This section describes the data used to generate the model, the model itself, the evaluation of the model against its alternatives, and the use of the model in creation of the connectivity estimate matrices.
It also includes background on the non-negative matrix factorization method used for decomposing the wild type connectivity matrix into latent factors.
Additional information about our data and methods are given in Supplemental Sections \ref{supp_sec:data} and \ref{supp_sec:methods}, respectively.

\newpage
\begin{figure}[H]
\subfloat[]{
\label{fig:mouse}
    \includegraphics[width=0.3\textwidth]{figs/figure1a.png}}
\subfloat[]{
\label{fig:injproj}
    \includegraphics[width=0.4\textwidth]{figs/inj_proj_figure_v2.png}}
\subfloat[]{
\label{fig:segment}
    \includegraphics[width=0.3\textwidth]{figs/fig1c.png}}
    \newline
 \subfloat[]{
 \label{fig:ontology}
    \includegraphics[width=0.35\textwidth]{figs/ontologyfigure.png}}
\subfloat[]{
 \label{fig:top}
    \includegraphics[width=0.35\textwidth]{figs/datasummary.png}}
\subfloat[]{
 \label{fig:combos}
    \includegraphics[width=0.35\textwidth]{figs/visp_counts.png}}   
    \caption{Experimental setting.  \ref{fig:mouse}  For each experiment, a Cre-dependent GFP-expressing transgene casette is transduced by stereotaxic injection into a Cre-driver mouse, followed by serial two-photon tomography imaging.
    \ref{fig:injproj} An example of the segmentation of projection (targets) and injection (source) for a single experiment. Within each brain (blue), injection (green) and projection (red) areas are determined via histological analysis and alignment to the Allen Common Coordinate Framework (CCF).
    \ref{fig:segment} Brain region parcellations within a coronal plane of CCFv3. \ref{fig:ontology} Explanation of nested structural ontology highlighting various levels of CCFv3 structure ontology. Lowest-level (leaf) structures are colored in blue, and structures without an injection centroid are colored in red.
    \ref{fig:top}  Abundances of tracer experiments by Cre-line and region of injection. \ref{fig:combos}  Co-occurrence of layer-specific centroids and Cre lines within VISp.}
    \label{fig:data}
\end{figure}

\newpage

\subsection{Data}

\skcomment{Update figure to show layer and leaf}

Our dataset $\mathcal D$ consists of $n=1751$ publicly available murine brain viral tracing experiments from the Allen Mouse Brain Connectivity Atlas.
Figure \ref{fig:mouse} summarizes the experimental process used to generate this data.
In each experiment, a mouse is injected with an adeno-associated virus (AAV) encoding green fluorescent protein (GFP) into a single location in the brain.
Location of fluorescence is mediated by the location of the injection, the characteristics of the transgene, and the genotype of the mouse.
In particular, Cre-driver mice are engineered to express Cre under the control of a specific and single gene promoter.
This localizes expression of Cre to regions with certain transcriptomic cell-types signatures.
In such Cre-driver mice, we used a double-inverted floxed AAV to produce fluorescence that depends on Cre expression in infected cells.
To account for the complex cell-type targeting induced by a particular combination of Cre-driver genotype and GFP promoter, we refer to the combinations of cell-types targeted by a particular combination of AAV and Cre-drive mice as cell-classes.
For example, we include experiments from Cre-driver lines that selectively label cell classes located in distinct cortical layers or other nuclei across the whole brain.
By our definition, wild type mice transduced with constitutitively active GFP promoters induce fluorescence of a particularly broad cell class.

For each experiment, the fluorescent signal imaged after injection is aligned into the Allen Common Coordinate Framework (CCF) v3, a three-dimensional average template brain that is fully annotated with regional parcellations \cite{Wang2020-po}.
The whole brain imaging and registration procedures described in detail in \citet{Oh2014-kh, Kuan2015-zz} produce quantitative metrics of fluorescence discretized at the $100 \; \mu$m \textbf{voxel} level. 
Given an experiment, this image was histologically segmented by an analyst into \textit{injection} and \textit{projection} areas corresponding to areas containing somas, dendrites and axons or exclusively axons of the transfected neurons.
An example of a single experiment rendered in 3D is given in Figure \ref{fig:injproj}.
Given an experiment $i$, we represent injections and projections as functions $x(i),y(i) : \mathcal B \to \mathbb R_{\geq 0}$, where $\mathcal B \subset [1:132] \times [1:80] \times [1:104]$ corresponds to the subset of the $(1.32 \times 0.8 \times 1.04)$ cm rectangular space occupied by the standard voxelized mouse brain.
We also calculate injection centroids $c(i) \in \mathbb R^3$ and regionalized projections $y_{\mathcal T} (i) \in \mathbb R^{T} $ given by the sum of $y(i)$ in each region.
A detailed mathematical description of these steps, including data quality control, is given in Supplemental Section \ref{supp_sec:dp}.

Our goal is the estimation of \textbf{regionalized connectivity} from one region to another.
A visual depiction of this region parcellation for a two-dimensional slice of the brain is given in Figure \ref{fig:segment}.
All structures annotated in the CCF belong to a hierarchically ordered ontology, with different areas of the brain are parcellated to differing finer depths within a hierarchical tree.
We denote the main levels of interest as major structures, summary structures, and layers.
Not every summary structure has a layer decomposition within this ontology, so we typically consider the finest possible regionalization - for example, layer within the cortex, and summary structure within the thalymus, and denote these structures as leafs.
As indicated in Figure \ref{fig:ontology}, the dataset used to generate the connectivity model reported in this paper contains certain combinations of region and cell class frequently, and others not at all.
A summary of the most frequently assayed cell classes and structures is given in Figures \ref{fig:top} and \ref{fig:combos}.
Since users of the connectivity matrices may be interested in particular combinations, or interested in the amount of data used to generate a particular connectivity estimate, we present this information about all experiments in Supplemental Section \ref*{supp_sec:data}.

\newpage

\subsection{Modeling Regionalized Connectivity}
\label{sec:modelling}

Cell-class specific connectivity $f:  \mathcal V \times \mathbb R^3 \times \mathbb R^3 \to \mathbb R_{\geq 0}$ gives the directed connection of a particular cell class from one position in the brain to another.
In contrast to \citet{Knox2019-ot}, which only uses wild type C57BL/6J mice, our dataset has experiments targeting $V = 114$ different combinations of Cre-driver mice and Cre-regulated AAV transgenes jointly denoted as $\mathcal V := \{v\}$.
As in \citet{Knox2019-ot}, we ultimately estimate an integrated regionalized connectivity defined with respect to a set of $S = 564$ source leafs $\mathcal S := \{ s\} $ and $T = 1123$ target leafs $\mathcal T := \{ t \}$, of which $1123 - 564  = 559$ are contralateral.
That is, we define
\begin{align*}
&\text{\textit {regionalized connectivity strength} } \mathcal C : \mathcal V \times \mathcal S \times \mathcal T \to \mathbb R_{\geq 0}  \text{ with } \mathcal C(v,s,t) = \sum_{l_{j} \in s} \sum_{l_{j'} \in  t} f(v,l_{j},l_{j'}), \\
&\text{\textit {normalized regionalized connectivity strength} } \mathcal C^N : \mathcal V \times \mathcal S \times \mathcal T \to \mathbb R_{\geq 0}  \text{ with } \mathcal C^N(v,s,t) = \frac{1}{|s|} \mathcal C(v,l_{j},l_{j'}), \\
&\text{\textit {normalized regionalized projection density} } \mathcal C^D : \mathcal V \times \mathcal S \times \mathcal T \to \mathbb R_{\geq 0} \text{ with } \mathcal C^D(v,s,t) = \frac{1}{| s | | t|}\mathcal C(v,l_{j},l_{j'})
\end{align*}
where $l_j$ and $l_{j'}$ are the locations of source and target voxels, and $|s|$ and $|v|$ are defined to be the number of voxels in the source and target structure, respectively.
Since the normalized strength and densities are computable from the strength via a fixed normalization, our main statistical goal is to estimate $\mathcal C (v,s,t) $ for all $v, s$ and $t$.% and density are
In other words, we want to estimate matrices $\mathcal C_v \in \mathbb R_{\geq 0}^{S \times T}$.
We call this estimator $\widehat { \mathcal C } $.

Construction of such an estimator raises the questions of what data to use for estimating which connectivity, how to featurize the dataset, what statistical estimator to use, and how to reconstruct the connectivity using the chosen estimator.
We represent these considerations as 
\begin{align}
\label{eq:estimator}
\widehat { \mathcal C }(v,s,t) = f^* (\widehat f (f_*( \mathcal D(v,s))).
\end{align}
This makes explicit the data featurization $f_{*}$, statistical estimator $\widehat f$, and any potential subsequent transformation $f^*$ such as summing over the source and target regions.
Denoting $ \mathcal D$ as a function of $v$ and $s$ reflects that we consider using different data to estimate connectivities for different cell-classes and source regions.
Table \ref{tab:estimators} reviews estimators used for this data-type used in previous work, as well as our two main extensions: the Cre-NW and \textbf{Expected Loss} (EL) models.
Additional information on these estimators is given in Supplemental Section \ref{supp_sec:estimators}.

\begin{table}[H]
    \centering
    \begin{tabular}{c|c|c|c|c|}
        Name & $f^*$ & $\widehat f$&  $ f_*$ & $\mathcal D(v,s)$ \\
        \hline
        NNLS \citep{Oh2014-kh} & $\widehat f (1_s)$ & \nnls(X,Y) & $X= x_{\mathcal S},Y = y_{\mathcal T}$ & $ I_m / I_m$ \\
        NW \citep{Knox2019-ot} &$ \sum_{l_s \in s} \widehat f (l_s)$ & \nw(X,Y)  & $X = l_s, Y = y_{\mathcal T}$ & $I_m /I_m$ \\
        Cre-NW& $\sum_{l_s \in s} \widehat f(l_s)$ & \nw(X,Y) & $X= l_s, Y = y_{\mathcal T}$  &$ (I_l \cap I_v) / I_m$ \\
        Expected Loss (EL) & $\sum_{l_s \in s} \widehat f (s)$ & $\el(X,Y,v)$ & $X= l_s, Y = y_{\mathcal T}, v$  &$I_l / I_m$
    \end{tabular}
    \caption{Estimation of $\mathcal C$ using connectivity data.
    The regionalization, estimation, and featurization steps are denoted by $f^*, \widehat f,$ and  $f_*$, respectively.
    The training data used to fit the model is given by index set $I$.
    We denote experiments with centroids in particular major brain divisions and leafs as $I_m$ and $I_l$, respectively.
    Data $I_l / I_m$ means that, given a location $l_s \in s \in m$, the model $\widehat f$ is trained on all of $I_m$, but only uses $I_l$ for prediction.
    The non-negative least squares estimator (NNLS) fits a linear model that predicts regionalized projection signal $y_{\mathcal T}$ as a function of regionalized injection signal $x_{\mathcal S}$.
    Thus, the regionalization step for a region $s$ is given by applying the learned matrix $\widehat f$ to the $s$-th indicator vector.
    In contrast, the Nadaraya-Watson model (NW) is a local smoothing model that generates a prediction for each voxel within the source structure that are then averaged to create estimate the structure-specific connectivity.
    }
    \label{tab:estimators}
\end{table}

Our contributions - the Cre-NW and Expected Loss (EL) models - have several differences from the previous methods.
In contrast to the non-negative least squares \citep{Oh2014-kh} and Nadaraya-Watson  \citep{Knox2019-ot} estimators that account only for source region $s$, our new estimators account cell class $v$, 
The Cre-NW estimator only uses experiments from a particular class to predict connectivity for that class, while the EL estimator shares information between classes within a structure.
Both of these estimator take into account both the cell-class and the centroid position of the experimental injection.
Like the NW and Cre-NW estimator, the EL estimator generates predictions for each voxel in a structure, and then sums them together to get the overall connectivity.
However, in contrast to the NW approaches, the EL estimate of the projection vector for a cell-class at a location weights the average projection of that cell-class in the region containing the location against the relative locations of all experimental centroids in the region regardless of class.
That is, cell-class and source region combinations with similar average projection vectors will be upweighted when estimating $\widehat f$.
Thus, all experiments that are nearby in three-dimensional space can help generate the prediction, even when there are few nearby experiments for the cell-class in question.
A detailed mathematical description of our new estimator is given in Supplemental Section \ref{supp_sec:el}.
\skcomment{cell-class versus Cre-line?}

\newpage

\subsection{Model evaluation}

We select optimum functions from within and between our estimator classes using \textbf{leave-one-out cross validation}, in which the accuracy of the model is assessed by its ability to predict projection vectors experiments excluded from the training data on the basis of their cell class and experimental centroid.
Equation \ref{eq:estimator} includes a deterministic step $f^*$ included without input by the data.
The performance of $\widehat {\mathcal C} (v,s,t)$ is thus determined by performance of $\widehat f (f_*(\mathcal D(v,s)))$.
Thus, we evaluate prediction of $f_{\mathcal T}: \mathbb R^3 \to \mathbb R_{\geq 0}^T$ - the regionalized connection strength at a given location.

Another question is what combinations of $v, s, $ and $t$ to generate a prediction for.
Our EL and Cre-NW models are leaf specific.
They only generate predictions for cell-classes in leafs where at least one experiment with a Cre-line targeting that class has a centroid.
To accurately compare our new estimators with less-restrictive models such as used in \citet{Knox2019-ot}, we restrict our evaluation set to Cre driver/leaf combinations that are present at least twice. 
The sizes of these evaluation sets are given in Supplemental Section \ref{supp_sec:model-evaluation}.

We use weighted $l2$-loss to evaluate these predictions.
\begin{align*}
\text{l2-loss } \ell (y_{\mathcal T}(i)),\widehat {y_{\mathcal T}(i))}) &:=   \| y_{\mathcal T} (i)) - \widehat {y_{\mathcal T}(i))} \|_2^2. \\
\text{weighted l2-loss } \mathcal L ( \widehat {f(f_*)}) &:= \frac{1}{|\{\mathcal S,\mathcal V\}|} \sum_{s,v \in \{\mathcal S,\mathcal V\}} \frac{1}{ |I_{s} \cap I_v |} \sum_{i \in (I_{s} \cap I_v ) } \ell (y_{\mathcal T}(i)), \hat f_{\mathcal T} (f_*(\mathcal D(v,s) \setminus i)) .
\end{align*}
This is a somewhat different loss from \citet{Knox2019-ot}, both because of the normalization of projection, and because of the increased weighting of rarer combinations of $s$ and $v$ implicit in the $\frac{1}{ |I_{s} \cap I_v |}$ term in the loss.
The establishment of a lower limit of detection and the extra cross-validation step used in the EL model to establish the relative importance of regionally averaged cell-class projection and injection centroid position are covered in Supplemental Section \ref{supp_sec:methods_lower}.

\newpage

\subsection{Connectivity analyses}

We examine latent structure underlying our estimated connectome using two types of unsupervised learning.
Our use of hierarchical clustering is standard, and so we do not review it here.
However, our application of non-negative matrix factorization (NMF) to decompose the estimated long-range connectivity into \textbf{connectivity archetypes} that linearly combine to reproduce the observed connectivity of some independent interest.
Non-negative matrix factorization refers to a collection of \textbf{dictionary-learning} algorithms for decomposing a non-negatively-valued matrix such as $\mathcal C $ into positively-valued matrices called, by convention, weights $W \in \mathbb R^{S \times q}_{\geq 0}$ and hidden units $H \in \mathbb R^{q  \times T}_{\geq 0}$.
Unlike PCA, NMF specifically accounts for the fact that data are all in the positive orthant.
The matrix $H$ is typically used to identify latent structures with interpretable biological meaning, and the choice of matrix factorization method reflects particular scientific subquestions and probabilistic interpretations. 

Our algorithm solves the following optimization problem
\begin{eqnarray*}
\label{eq:nmf}
\nmf(\mathcal C, \lambda, q) := \arg \min_{W\in \mathbb R^{S \times q}_{\geq 0}, H \in \mathbb R^{q  \times T}_{\geq 0}} \frac{1}{2}\| 1_{d(s,t) > 1500 \mu m} \odot \mathcal C - WH\|_2^2  + \lambda  (\|H \|_1 + \|W \|_1) .
\end{eqnarray*}
For this decomposition we ignore connections between source and target regions less than  $1500 \mu m$ apart.
This is because short-range projections resulting from diffusion dominate the matrices $\hat {\mathcal C}$, and represent a less-interesting type of biological structure.
We set $\lambda = 0.002$ to encourage sparser and therefore more interpretable components.
We use unsupervised cross-validation to determine an optimum $q$, and show the top $15$ stable components \citep{Perry2009-ia}.
Stability analysis accounts for the difficult-to-optimize NMF program by clustering the resultant $H$ from multiple replicates.
Since the NMF objective is difficult to optimize and sensitive to initialization, we follow up with a stability analysis.
The medians of the component clusters appearing frequently across NMF replicates are selected as \textbf{connectivity archetypes}.
Details of these approaches are given in Supplementary Sections \ref{supp_sec:matrix_factor_methods} and \ref{supp_sec:matrix_factor_results}.
\newpage
\section{Results}
\label{sec:results}

Our results include evaluation of model fit, the Cre-specific connectivity matrices themselves, and retrospective analyses of these matrices for  patterns related to cre-type and source and target regions.

\subsection{Model evaluation}
\label{sec:model_eval}

%Note that this is also the largest evaluation set for 2-stage model, since leave-one-out means of each cre-line may only be computed for lines which are present at least twice in the leaf.

%Clustered source-cell combinations have similar projection patterns.

%In particular, we illustrate which source and cell-type combinations behave similarly, and which target projections tend to co-occur. 

Table \ref{tab:crossvalidation} contains weighted losses from leave-one-out cross-validation of candidate models.
Our EL model generally performs better than the other Nadaraya-Watson estimators that we consider.
For example, the NW Major-WT model is the model from  \citet{Knox2019-ot}.
The EL model combines the good performance of class-specific models like NW Leaf-Cre in regions like Isocortex with the good performance of class-agnostic models in regions like Thalymus.
Additional information on model evaluation, including class and structure specific performance, is given in Appendix \ref{supp_sec:model-evaluation}
In particular, Supplementary Table \ref{tab:loss} contains the sizes of these evaluation sets in each major structure, and Supplementary Section \ref{supp_sec:loss_subsets} contains the structure- and class specific losses.

\begin{table}[H]
\small
\begin{tabular}{lrrrrrrr}
\toprule
& Mean Leaf-Cre & NW Major-Cre& NW Leaf-Cre & NW Leaf &NW Major-WT  & NW Major & EL \\
$\widehat f$ &           Mean & \multicolumn{5}{l}{NW} &     EL \\
$\mathcal D$ & $I_c \cap I_L$ & $I_c \cap I_M$ & $I_c \cap I_L$ & $I_L$ & $I_{wt} \cap I_M$ &  $I_M$ &  $I_L$ \\
\midrule
Isocortex &          0.264 &          0.256 &          0.257 &             0.358 &          0.370 &  0.370 &  \textbf{0.246} \\
OLF       &          0.185 &          0.215 &          0.184 &             \textbf{0.131 }&          0.175 &  0.175 &  0.136 \\
HPF       &          0.176 &          0.335 &          0.170 &             0.201 &          0.235 &  0.235 &  \textbf{0.148} \\
CTXsp     &         \textbf{ 0.758} &          \textbf{0.758} &          \textbf{0.758} &             \textbf{0.758 }&          \textbf{0.758 }&  \textbf{0.758 }&  \textbf{0.758} \\
STR       &          0.131 &         \textbf {0.121} &          0.129 &             0.173 &          0.236 &  0.236 &  0.125 \\
PAL       &          0.220 &          0.223 &          0.220 &             0.339 &          0.324 &  0.324 &  \textbf{0.197} \\
TH        &          0.634 &          0.626 &          0.634 &             0.362 &         \textbf {0.360} &  \textbf{0.360 }&  0.366 \\
HY        &          0.388 &          0.392 &          0.381 &             0.359 &          0.338 &  0.338 &  \textbf{0.331} \\
MB        &          0.213 &          0.232 &          0.201 &             0.276 &          0.285 &  0.285 &  \textbf{0.195} \\
P         &          0.309 &          0.309 &          0.309 &             0.404 &          0.402 &  0.402 &  \textbf{0.306} \\
MY        &          0.261 &          0.340 &          0.261 &             0.188 &         \textbf{ 0.187 }&  \textbf{0.187} &  0.198 \\
CB        &          0.062 &          \textbf{0.061} &          0.062 &             0.067 &          0.111 &  0.111 &  0.068 \\
\bottomrule
\end{tabular}
\label{tab:crossvalidation}
\caption{Losses from leave-one-out cross-validation of candidate models. \textbf{Bold} numbers are best for their major structure.}
\end{table}

%Additional information on model evaluation, including class and structure specific performance, is given in Appendix \ref{supp_sec:model-evaluation}

\newpage

\subsection{Connectivities}

Our main result is the estimation of matrices $\hat {\mathcal C}_v$ representing connections of source structures to target structures for particular cre-lines $v$. 
We exhibit several characteristics of interest, and confirm the detection of several well-established connectivities within our tensor.
Many additional interesting biological processes are visible within this matrix - more than we can report in this paper - and it is our expectation that these will be identified by users of our results.
The connectivity tensor and code to reproduce it are available at \url{https://github.com/AllenInstitute/mouse_connectivity_models/tree/2020}.

\subsubsection{Overall connectivity}

The connectivity matrix for wild-type connectivities from leaf sources to summary structure targets is illustrated in Figure \ref{fig:connectome}.
Several expected biological processes are evident.
For example, intraareal connectivities are clear, as are ipsilateral connections between cortex and thalymus.
The clear intraareal connectivities mirror previous estimates in \citet{Oh2014-kh} and \citet{Knox2019-ot} and descriptive depictions of individual experiments in \citet{Harris2019-mr}.
Compared with \citet{Knox2019-ot}, our more discretized source smoothing and greater number of experiments leads to a significantly more discretized connectivity matrix.
%This is generally expected - for example, different cortical layers have different connectivities.

\newpage

\begin{figure}[H]
\centering
    \subfloat[] {
    \label{fig:full_wt}
   % \raisebox{-.5\height}
    \includegraphics[width = \textwidth]{../../analyses/paper/figures/conn_leafs_0617.png}
    } 
        \newline
       \subfloat[] {
    \label{fig:cortex_wt}
   % \raisebox{-.5\height}
    \includegraphics[width = .5\textwidth]{../../analyses/paper/figures/conn_sum_cortex_0617.png}
    } 
   \caption{Wild-type connectivities.
   \ref{fig:full_wt} Log wild-type connectivity matrix $\log \mathcal {C} (s,t,v_{wt})$.
   \ref{fig:cortex_wt} Log wild-type intracortical connectivity matrix at the summary structure level.}
   \label{fig:connectome}
\end{figure}



\newpage
\subsubsection{Class-specific connectivities}

We have generated $V = 114$ cell-class specific structural connectivities $\mathcal C(v) \in \mathbb R^{S \times T}$.
Exhaustive comparison of this estimated behavior is prohibitive, but we do exhibit several examples of our class specific connectivities conforming to well-known behaviors.
These validation cases are given in Figure \ref{fig:data_ct}.

We begin by plotting subsets of the estimated connectivities in the well-studied VISp and MO regions in Figure \ref{fig:ct_spc}.
The localization of Rbp4-Cre and Ntsr1-Cre injection centroids to layers 5 and 6 respectively is evident (see also Supplemental Figure \ref{fig:iso_count}). 
These layers project to their expected targets \citet{Jeong2016-dc}.
In VISp, the Ntsr1-Cre line strongly targets the thalymic LP nuclei, and in MO, layer 5 projects to anterior basolateral amygdala (BLA) and capsular central amygdala (CEA), while layer 6 does not.

Figure \ref{fig:ct_clust} shows a collection of connectivity strengths generated using cre-specific models for wild-type, Cux2, Ntsr1, Rbp4, and Tlx3 cre-lines from visual signal processing leafs in the cortex to cortical and thalymic nucleii.
This shows that cell-class has a dominating effect on projection in certain regions.
We use heirarchical clustering to sort source structure/cell-class combinations by the similarity of their structural projections, and sort target structures by the structures from which they receive projections.
Examining the former, we can see that the Ntsr1 Cre-line distinctly projects to thalymic nucleii, regardless of summary structure.
This contrasts with the tendency of other cell classes to project intracortically in a manner determined by the source structure.
Similarly, layer 6 targets are not strongly projected to by any of the displayed Cre-lines.
There are too many targeted summary structures to plot here, but we expect that the source profile of each target clusters by structure.

\newpage

\begin{figure}[H]
\subfloat[]{
\label{fig:ct_spc}
    \includegraphics[width=.8\textwidth]{../../analyses/paper/figures/visp_mo_0615.png} 
    }
    \newline
 \subfloat[]{
 \label{fig:ct_clust}
    \includegraphics[width=.8\textwidth]{../../analyses/paper/figures/heirarchical.png}
    }
    
    \caption{ Cell-class and layer specific connectivities from VISp and MO.
    		This figure shows a preselected subset of putatively interesting connectivities from VISp and MO.
    		\ref{fig:data_ct}
		Heirarchical clustering of connectivity strengths from visual signal processing cell-types to cortical and thalymic targets.
		Cre-line, summary structure, and layer are labelled on the sources.
		Major brain division and layer are labelled on the targets.
		Note that sources/cre combinations are only included if there is at least one experiment of that cre-line in that particular leaf.}
\label{fig:data_ct}
\end{figure}

\newpage

\subsection{Connectivity Analyses}

Each structural connectivity matrix is a high-dimensional representation of relatively few biological processes.
As discussed in \citet{Knox2019-ot}, one of the most basic processes underlying the observed connectivity is the tendency of each source region to predominantly project to proximal regions.
For example, the heatmap in \ref{fig:dist_bw_str} shows intraregion distances clearly contains an overall pattern reminscent of the connectivity matrix in \ref{fig:connectome}.
These connections are biologically meaningful, but also unsurprising, and their relative strength biases learned latent coordinate representations away from long-range structures.
For this reason, we establish a $1500 \mu m$ 'distal' threshold within which to exclude connections for our analysis.

Perhaps more interestingly in our setting, certain cell-types and layers have a characteristic connectivity pattern.
We therefore perform non-negative matrix factorization on distal wild-type connectivities to estimate these characteristic patterns in a probabilistic way.
This decomposes the remaining censored connectivity matrix into a relatively small number of distinct signals.
These signals are plotted in Figure \ref{fig:nmf_results}, and technical details and intermediate results are given in Supplemental Sections \ref{supp_sec:matrix_factor_methods} and \ref{supp_sec:matrix_factor_results}, respectively.

The plotted decomposition shows that these underlying connectivity archetypes correspond strongly to major brain division.
However, certain components that predominantly represent connectivity from a given major brain division may also be accessed from other areas.
For example, the IP and FN regions of CB are strongly associated in \ref{fig:W} with the component projecting to MY in \ref{fig:H}.
%The overall wild-type connectivity strength matrix also displays an underlying modelable structure.

%This relationship is plotted in \ref{fig:nmf} b), showing that there exists substantial variability that would be impossible to model with low-error in a univariate model, even using the diffusion model suggested in \citet{Knox2019-ot}.
%We then apply non-negative matrix factorization (NMF) to 


%, and apply an unsupervised cross-validation method to select the optimum number of signals %\skcomment{Percent error... show reconstruction? log scale?}.

\newpage

\begin{figure}[H]
\begin{tabular}[t]{c}
\subfloat[]{\includegraphics[width = .8 \textwidth]{../../paper/figures/H_wt_0617.png}
\label{fig:H}}
\\
%\subfloat[]{\includegraphics[width = 1.5in]{figs/figsforpres/test_train.png}} &
\subfloat[]{
\includegraphics[width = .3\textwidth]{../../paper/figures/W_wt_0617.png}
\label{fig:W}
}
\end{tabular}
\caption{Non-negative matrix factorization results $\mathcal C_{wt} = WH$ for $q = 15$ components.
\ref{fig:H} Latent space coordinates $H$ of $\mathcal C$.
Target major structure and hemisphere are plotted.
\ref{fig:W} Loading matrix $W$.
Source major structure and layer are plotted.
}
\label{fig:nmf_results}
\end{figure}

\newpage


%This analysis shows that there are relatively few architypal signals underlying the overall connectome.

%simply exhibit a collection of these patterns using several factorization methods applied to the connectivity matrix $\mathcal C$.

%Association of these patterns with the underlying biological processes is a fundamental goal in neuroscience.
%Since much of the connectivity matrix can be predicted solely based off of location information.  For this reason, we subtract our the simple $\hat f_{d} c(p_1), c(p_2)$ where $\hat f$ is the estimated relation of distance between regional centroids and connectivity strength.

%, which we elucidate through n. First, projection signals cluster by target and by source; we can identify similarly behaving structures and neural targets that tend to co-occur.  Second, the specific cell-types targeted by the various cre-lines themselves generate a reduced-dimension space. Under the assumption that cell-type determines projection pattern, we can investigate which cell-types are present in which of the projecting structures.

\newpage

\begin{comment}
%}
% \begin{figure}[h]
%     \centering
%     \includegraphics[width = 6in]{figs/Figure5.png}
%     \caption{a) Distances between structures \skcomment{Add units to distance legend}. b) Relation between connectivity strength and distance. c) Train error and test error of non-negative matrix factorization of non-local ($> 1500 \mu m$) connectivities. d) The top $15$ factorization components. \skcomment{Why is compressability so high}}
%     \label{fig:nmf}
% \end{figure}
% \newpage





%Our core model evaluation method is leave-one-out cross validation. This method is robust to the trivial overfitting in Nadaraya-Watson bandwidth selection. We show that incorporation of cre lines improves model performance in the following experimental set-ups. This requires at least two experiments in every experimental division, even in the 2-stage model, since leave-one-out means of each cre line may only be computed for cre-lines which are present at least twice in the leaf. This is the set up for a model trained on all data. However, for alternate models, such as the major-structure divided model from \citet{Knox2019-ot}, the potential evaluation set is larger. In order to compare between methods, we therefore restrict to the smallest set of evaluation indices, which is to say, virus-leaf combinations that are present at least twice.  This means that in some cases, our training set exceeds our evaluation set in size. 

 \begin{comment}
\begin{figure}[H]
\centering
    \subfloat[] {
   % \raisebox{-.5\height}
    \includegraphics[width = \linewidth]{../../analyses/paper/figures/conn_leafs_0617.png}
    } 
    \\
    \vspace{-2cm}
    \subfloat[]{
    \adjustbox{valign=c}{
    \tiny
\begin{tabular}{lrllll}
\toprule
{} &  \#\pbox{20cm} {Ipsilateral \\ Leaf Targets } & \pbox{20cm}{Top \\ Entropy }& Bottom Sparsity & Bottom Entropy & Top Sparsity \\
\midrule
Isocortex &                          51 &          CP &             BAC &            BAC &         ENTl \\
OLF       &                          11 &         TMv &             III &            III &          NaN \\
HPF       &                          15 &          IG &             EPv &             PA &          NaN \\
CTXsp     &                           7 &          TT &              FC &            APr &           TT \\
STR       &                          14 &         RPA &             ISN &            PYR &           TU \\
PAL       &                           9 &          PG &           ACVII &             GR &           MG \\
TH        &                          44 &         NOD &              DN &         SSp-ll &          SCm \\
HY        &                          44 &         CLA &              SH &            LSc &           DG \\
MB        &                          39 &         NDB &            SubG &            SGN &          SUB \\
P         &                          26 &          MT &            Acs5 &            SOC &          NDB \\
MY        &                          43 &          RT &             NaN &             OV &          EPd \\
CB        &                          18 &         ECT &             AOB &            MOB &           GU \\
\bottomrule
\end{tabular}
    }
}
    \subfloat[]{
    \adjustbox{valign=c}{
    \includegraphics[width = 3in]{figs/st_graph}
    }
    } \\
     \vspace{-4cm}
    \subfloat[]{
    \adjustbox{valign=c}{
    \tiny
    \begin{tabular}{lrllll}
       
\toprule
{} &  \# Ipsilateral Leaf Targets & Top Entropy & Bottom Sparsity & Bottom Entropy & Top Sparsity \\
\midrule
Isocortex &                          51 &          CP &             BAC &            BAC &         ENTl \\
OLF       &                          11 &         TMv &             III &            III &          NaN \\
HPF       &                          15 &          IG &             EPv &             PA &          NaN \\
CTXsp     &                           7 &          TT &              FC &            APr &           TT \\
STR       &                          14 &         RPA &             ISN &            PYR &           TU \\
PAL       &                           9 &          PG &           ACVII &             GR &           MG \\
TH        &                          44 &         NOD &              DN &         SSp-ll &          SCm \\
HY        &                          44 &         CLA &              SH &            LSc &           DG \\
MB        &                          39 &         NDB &            SubG &            SGN &          SUB \\
P         &                          26 &          MT &            Acs5 &            SOC &          NDB \\
MY        &                          43 &          RT &             NaN &             OV &          EPd \\
CB        &                          18 &         ECT &             AOB &            MOB &           GU \\
\bottomrule
\end{tabular}
}
}
\vspace{-4cm}
    \subfloat[]{
    \adjustbox{valign=c}{
    \includegraphics[width = 3in]{figs/page_graph}
    }
    }
\end{figure}
\end{comment}

%\begin{figure}[p]
%    \centering
%    \includegraphics[width = 18cm]{figs/visp_mo.png}
%    \caption{Cre-line specific connectivity matrices for a selection of sources and targets are displayed as heatmaps. Sources without a injection of that cre-type are blank.}
%    \label{fig:my_label}
%\end{figure}



\newpage


\section{Discussion}

Flattening $\mathcal C$ prior to unsupervised analysis is not necessarily recommended, but provides an easy solution for this problem.

With respect to the model, a Wasserstein-based measure of injection similarity per structure would combine both the physical simplicity of the centroid model while also incorpating structural knowledge.

The Nadaraya-Watson weighting procedure introduced here is, to our knowledge, novel.  In particular, our method of utilizing the expected loss to weight points differs from the minimization task of fitting data to weighted sums of neighbors \citep{Saul2003-th}.  We make a key assumption: that the additional statistical accuracy of including more samples makes up for the fact that their expected accuracy is lower. Note that this assumption can be easily violated, if, for example, the data is distributed on a circle without error, and only nearest neighbors are most predictive.

Model averaging based off of cross-validation has been implemented in \citet{Gao2016-qe}, but we note that our approach makes use of a non-parametric estimator, rather than an optimization method for selecting the weights.  \skcomment{CITE METHOD THAT SELECTS WEIGHTS IN KERNEL (has catchy name)}

%and is related to the theoretical research problem of interpreting learned dynamical systems. 



%We synthesize the two equivalently in a product space as
%\begin{eqnarray*}
%d(i,i')^2 = \|\mu (v(i)) - \mu (v(i')\|_2^2 + \|c(i) - c(i')\|_2^2.
%\end{eqnarray*}

%However, the weighting of the two feature types is unclear. In lieu of an optimization-based approach for learning the weights of each distance coordinate, we utilize the following approach.

%(indeed, only wild-type tracing experiments were 


%where
%\begin{equation}
%f: \mathbb R^3 \times \mathcal V \to \mathbb R^{S_{IMC}}
%\end{equation}
%is the projection of a particular voxel.


%We therefore write the generic model with errors-in-variables as
%The general model is thus

%\begin{eqnarray*}
%y = f(x + \epsilon_x, v) + \epsilon_y.
%\end{eqnarray*}

%We assume $\epsilon_x = 0$, and $x$ and $y$ have been regionalized as $r_y(y(i)) \in \mathbb R^{S_{IMC}}, r_x(x(i) \in \mathbb R^S$. \citet{Knox2019-ot} utilizes the model $r(y(i)) = f_{NW} (c(x(i)))$, where $f_{NW}$ is the Nadaraya-Watson estimator.  In contrast, \citet{Oh2014-kh} uses the model $r(y(i)) = f_{NNLS} (r(x(i)))$ where $f_{NNLS}$ is fitted non-negative least squares.  In both cases, data consisted of only the wild type cre line.
%Although the Knox model is conceptually simpler, it generated improved predictive performance. This improvement is due to the fact that the regionalization map $r$ is a relatively coarse description of the injection $x(i)$, compared with the centroid function $c(x(i))$.  Furthermore, the slow convergence of the Nadaraya-Watson estimator is mediated by the lower dimensionality of $c(x(i))$ compared with $r_x x((i))$, and generalization issues caused by poor-conditioning of the somewhat sparse design matrix $r(x(1:n))$ are avoided.

%The major-structure specific Nadaraya-Watson estimator from $\citet{Knox2019-ot}$ nevertheless appears poorly adapted for our circumstances since it does not take into account the virus $v$. It is not straightforward to include the viral strain in either $f_{NW}$ or $f_{NNLS}$. For $f_{NW}$, it is not clear how a one-hot encoded class-membership feature should be weighted. For $f_{NNLS}$, our sample size seems too low to utilize a fixed or mixed effect, particularly since the impact of the virus depend on the particular injection region, motivating use of an interaction term.

%A related, more simplistic, model is to average exactly in the cre-leaf combination. \begin{eqnarray*}
%y = f_1(x) = \mu_v \\
%\end{eqnarray*}

%This average model is in the limit of both the Nadaraya-Watson and non-negative least square models. In the former, the bandwidth is expanded infinitely. In the later, the design matrix is diagonal in centroid location in a block diagonal expanded design matrix only filled in the appropriate cre block.

%Although both $f_{NW}$ and $f_{VL}$ work well, we are still have not yet estimated when one is preferable to another. This trade-off is important to understand, and even a misspecified model may be useful, since the sample-size to noise ratio is low. We do this by relating difference in creline and distance between centroids to the loss using the following simple non-parametric estimator.

%The benefit of this model's simplicity is that it is immediately apparent that it defines a refeaturization of the one-hot matrix in terms of the leave-one-out means of each class. We can thus define an asymptotically/out-of-sample (although asymmetric in leave-one-out) distance between points  


%In order to measure the accuracy of our estimate $\hat {\mathcal C}$, we assess our model via cross-validation of held-out experimental projections.

%\begin{eqnarray*}
%\hat \epsilon = \mu_v - \mu_{v'} \sim y - y'
%\end{eqnarray*}


%Another way of conceptualizing the Knox model is as a residual model with regionalized features only including the centroid region.  This is precisely an average model.


%On the other hand, we have 
%\begin{eqnarray*}
%\hat \epsilon = f(c - c') \sim y - y'
%\end{eqnarray*}





%\begin{algorithmic}
%\label{alg:model}
%\begin{algorithm}{Injection centroids $c(1:n)$, normalized projections $n(r(y(1:n)))$, viruses $v(1:n)$. target centroid $c$, target virus $v$}
%\STATE Get structures $s(1:n) = r(c(1:n))$, $s = r(c)$
%\STATE Target encode $v(1:n)$ and $v$ with $n(r(y(1:n)))$
%\STATE Estimate expected loss $l_{ii'} = \hat f (\|c(i) - c(i')\|_2^2, \|t(v(i)) - t(v(i'))\|_2^2)$
% \RETURN $\tilde y(i)$ (optional $\tilde x(i)$ )
%\end{algorithm}
%\end{algorithmic}

%%%%%%%%%%%%%%%%%%%%%%%%%%%%%%%%%%%%%%%%%%%%%%%%%%%%%%%%%%%%%%%%%
%%% End of Article

% \acknowledgments
% \section{Supporting Information} (optional)
% \section{Competing Interests} (optional)
% \bibliography{<name of .bib file>}




\newpage

\acknowledgments
The Funder and award ID information you input at submission will be introduced by the publisher under a Funding Information head during production.
Please use this space for any additional acknowledgements and verbiage required by your funders.
\newpage
\section{Supplemental Information}

\subsection{Data}
\label{supp_sec:data}

This section describes the set of leaf and cre-line experimental combinations.

\newpage

\begin{figure}[H]
    \centering
    \includegraphics[width = 7in]{figs/CB centroid densityoct12.png}
    \label{fig:my_label}
\end{figure}
\newpage

\begin{figure}[H]
    \centering
    \includegraphics[width = 7in]{figs/TH centroid densityoct12.png}
    \caption{Caption}
    \label{fig:my_label}
\end{figure}
\newpage

\begin{figure}[H]
    \centering
    \includegraphics[width = 7in]{figs/CTXsp centroid densityoct12.png}
    \caption{Caption}
    \label{fig:my_label}
\end{figure}
\newpage

\begin{figure}[H]
    \centering
    \includegraphics[width = 7in]{figs/P centroid densityoct12.png}
    \label{fig:my_label}
\end{figure}
\newpage

\begin{figure}[H]
    \centering
    \includegraphics[width = 7in]{figs/PAL centroid densityoct12.png} 
    \label{fig:my_label}
\end{figure}
\newpage

\begin{figure}[H]
    \centering
    \includegraphics[width = 7in]{figs/Isocortex centroid densityoct12.png}
    \label{fig:my_label}
\end{figure}
\newpage

\begin{figure}[H]
    \centering
    \includegraphics[width = 7in]{figs/MB centroid densityoct12.png} 
    \label{fig:my_label}
\end{figure}
\newpage

\begin{figure}[H]
    \centering
    \includegraphics[width = 7in]{figs/MY centroid densityoct12.png} 
    \label{fig:my_label}
\end{figure}
\newpage

\begin{figure}[H]
    \centering
    \includegraphics[width = 7in]{figs/HPF centroid densityoct12.png} 
    \label{fig:my_label}
\end{figure}
\newpage

\begin{figure}[H]
    \centering
    \includegraphics[width = 7in]{figs/OLF centroid densityoct12.png} 
    \label{fig:my_label}
\end{figure}
\newpage

\begin{figure}[H]
    \centering
    \includegraphics[width = 7in]{figs/STR centroid densityoct12.png} 
    \label{fig:my_label}
\end{figure}
\newpage

\begin{figure}[H]
    \centering
    \includegraphics[width = 7in]{figs/HY centroid densityoct12.png} 
    \label{fig:my_label}
\end{figure}

\newpage
\subsection{Structure information}

\begin{figure}[H]
    \centering
    \includegraphics[width = 7in]{../../analyses/paper/figures/distances_leafs.png} 
    \label{fig:distances}
    \caption{Distance between structures.  Short-range connections are masked in red}
\end{figure}

\subsection{Model evaluation}
\label{supp_sec:model-evaluation}


\begin{table}[H]
\begin{tabular}{lrrrr}
\toprule
{} &  Total &  Cre-Summary &  Cre-Summary, Leaf &  Cre-Leaf \\
\midrule
0  &     36 &           10 &                  9 &         4 \\
1  &      7 &            2 &                  2 &         2 \\
2  &    122 &           79 &                 79 &        62 \\
3  &     85 &           41 &                 41 &        41 \\
4  &   1128 &          838 &                829 &       732 \\
5  &     68 &           23 &                 23 &        18 \\
6  &     46 &            7 &                  7 &         7 \\
7  &     35 &           17 &                 17 &        17 \\
8  &     33 &            8 &                  8 &         8 \\
9  &     30 &           11 &                 11 &        11 \\
10 &     78 &           45 &                 45 &        45 \\
11 &     83 &           29 &                 29 &        29 \\
\bottomrule
\end{tabular}
\caption{Number of experiments available to evaluate models in leave-one-out cross validation. 
Models that rely on a finer granularity of modeling have less data available to validate with.}  
\end{table}

\newpage 
\section{Supplemental methods}
\label{sec:supp_methods}

This section consists of additional information on preprocessing of the neural connectivity data, estimation of connectivity, and matrix factorization.

\subsection{Data preprocessing}
\label{sec:dp}

%Injection mask
%Projection mask
%Data quality mask
%From this, a weighted centroid $c(x_i) \in \mathbb R^3$ and 
%Regionalization occurs in several ways.
%A per-region average $r(x_i) \in \mathbb R^{M_i}$ may be computed, where $M_i$ is the number of substructures in the major brain region.
%Similarly, for the projection, $r(y_i) \in R^{M}$.

Several data prepreprocessing steps take place prior to evaluations of the connectivity matrices.
Injections and projections were downloaded using the Allen SDK.
These were originally annotated manually.
The injection and projection vectors are Hadamard multiplied by a data quality matrix. We also have a map $A: \mathbb R \to \mathbb R^{|S|}$ where $S$ is the number of structures that takes the average value for voxels in that structure

\begin{algorithmic}
%\label{alg:norm}
\begin{algorithm}{Injection $x(i)$, Projection $y(i)$, Injection centroid $c(i) \in \mathbb R^3$, injection fraction $F(i)$, data mask $M(i)$}
\State Data-quality censor $y_M (i) = M(i) \odot y(i)$ (also $x_M(i) = M(i) \odot x(i)$ for linear model)
%\State Normalize $x_N(i) = F(i) \odot x_M(i)$ (only for linear model)
%\State Average structures $y_S (i) = A y_M(i)$ (also $x_S (i) = A x_M(i)$ for linear model)
%\State Normalize $\tilde y(i) = \frac{y_S (i)}{\|y_S (i)\|}$ (also $\tilde x (i) = A x_M(i)$ for linear model)
% \Return $\tilde y(i)$ (optional $\tilde x(i)$ )
\end{algorithm}
\end{algorithmic}

The data-quality censor is established by \skcomment{fill}
The injection fraction accounts for the relatively coarse graining of the voxel grid compared with the histological analysis used to establish the injection region.  In particular, certain voxels are only partially contained within the injection region.


One basic but significant methodological change from \citet{Knox2019-ot} is the normalization of projection vectors.

\begin{comment}
\begin{figure}
    \centering
    \includegraphics{}
    \caption{Caption}
    \label{fig:my_label}
\end{figure}
\end{comment}
The loss function in \citet{Knox2019-ot} is
\begin{eqnarray*}
\frac{\|y - \hat y\|} {\|y\|\|\hat y\|}
\end{eqnarray*}

%We find the $l2$ norm not appropriate in a few ways 1) there is not a clear inner product structure on the connectome.
%doesn't matter if we normalize prior to regionalization as long as we don't correct for density of the target region.
%which, due to normalization, can be rewritten as simply the $l2$ loss $\|y - \hat y\|.$  Note that, due to normalization, $\|y - \hat y\| = \langle y - \hat y , y - \hat y \rangle = 2 - 2 \langle y, \hat y\rangle $, and, applying normalization again, $ \langle y, \hat y\rangle $ is the vector cosine of $\hat y$ and $y$.

\newpage
\subsection{Estimators}

Our estimators span a range of training and featurization methods.
One commonality is that they model a connectivity vector $f (\mathcal D, v,s)  \in \mathbb R^T$, and so we may write
\begin{eqnarray*}
f (v,s,t) = f (v,t)[t].
\end{eqnarray*}
Thus, for the remainder of this section, we will discuss only $f (s,v)$.

\subsubsection{Centroid-based Nadaraya-Watson}

In the Nadaraya-Watson approach of \citet{Knox2019-ot}, the injection is considered only through its centroid, while the projection is considered regionalized.
%The prediction for a given region is then given by the integral of predictions over that region, which is computed as a sum over voxels.
That is,
\begin{eqnarray*}
f_*({\mathcal D}_i) = \{c(x_i) , r(y_i)\}.
\end{eqnarray*}
Since the injection is considered only by its centroid, this model only generates predictions for particular locations $c$, and the prediction for a structure $s$ is given by integrating over locations within the structure
\begin{eqnarray*}
\label{eq:regionalize}
f^* (\hat f (f_*(\mathcal D))) (v,s) = \sum_{c \in s} \hat f (f_*(\mathcal D)) (v,c),
\end{eqnarray*}
This $\hat f$ is the Nadaraya-Watson estimator
\begin{eqnarray*}
\hat f_{NW}( c(x_{1:n}) , r(y_{1:n}) ) (c,v) :=  \sum_{i \in I} \frac{ \omega_{c(x_i) c}}{\sum_{i \in I} \omega_{c(x_i), c}} r(y_i)
\end{eqnarray*}
where $\omega_{c(x_i) c} = \exp( - \gamma d( c , c(x_i))^2 )$ and $d$ is the Euclidean distance between centroid $c(x_i)$ and voxel $c$.
Several facets of the estimator are visible here. %$f_{NW}^{\gamma}$ is the Nadaraya-Watson estimator with smoothing given by inverse-bandwidth $\gamma$.
A smaller $\gamma$ corresponds to a greater amount of smoothing, and index set $I \subseteq  \{1:n\}$ indicates which experiments to use to generate the prediction.
Fitting $\gamma$ via empirical risk minimization therefore bridges between $1$-nearest neighbor prediction and averaging of all experiments in $I$.
In \citet{Knox2019-ot}, $I$ consisted of experiments sharing the same brain division.
 Restricting of index set to only include experiments with the same neuron class gives the class-specific model.

\newpage
\subsubsection{The expected-loss estimator}

\label{supp_sec:el}

The response induced by each of the cre-lines is effected by both the injection location and the targeted cell types.
Cre-lines that target similar cell types are therefore expected to induce similar projections, and including similar cre-lines in our estimator thus increases the effective sample size.
In order to leverage this fact in a data-driven way, we introduce an estimator that assigns a predictive weight to each training point that depends both on its centroid-distance and cre-line.
This weight is determined by the expected prediction error of each of the two feature types, as determined by cross-validation.
These weights are then utilized in a Nadaraya-Watson estimator in a final prediction step.

%
%standard optimization approach gives plane
We formalize cre-line behavior as the average regionalized projection of a cre-line in a given leaf.
This vectorization of categorical information is known as target encoding.
We define a {\textit cre-distance} in a leaf to be the distance between the target-encoded projections of two cre-lines.
The relative predictive accuracy of cre-distance and centroid distance is determined by fitting a surface of projection distance as a function of cre-distance and centroid distance. 


%centroids are weighted more highly in the centroid-based Nadaraya-Watson estimator.
%this estimator is an average (limit) of the cre-specific NW

In mathematical terms, our full feature set consists of the centroid coordinates and the target-encoded means of the combinations of virus type and injection-centroid structure.
That is, 
\begin{eqnarray*}
f_*({\mathcal D}_i) = \{c(x_i) , \bar r(y_{I_v}), r(y_i) \}.
\end{eqnarray*}
$f^*$ is defined as in \eqref{eq:regionalize}. The expected loss estimator is then 
\begin{eqnarray*}
\hat f_{EL} (c, c(x_i),v , r(y_{I_v})) =  \sum_{i \in I} \frac{ \nu {(c(x_i) , c, v_i, v)}}{\sum_{i \in I} \nu {(c(x_i) , c, v_i, v) }} r(y_i)
\end{eqnarray*}
where
\begin{eqnarray*}
\nu_i = \exp (- \gamma g( d(c, c(x_i))^2, d(\bar r (v), \bar r (v_i))^2))
\end{eqnarray*}
Note that $g$ must be a concave, non-decreasing function of its arguments with with $g(0,0) = 0$, then $g$ defines a metric on the product of the metric spaces defined by experiment centroid and target-encoded cre-line, and $\hat f_{EL}$ is a Nadaraya-Watson estimator.  A derivation of this fact is given in Appendix \ref{supp_sec:el}, and we therefore use shape-constrained B-splines to estimate $g$.

This contrasts with the model is \citet{Knox2019-ot}, where $\hat f(c)$ does not depend on $v$, and \citet{Oh2014-kh}, where connectivity was directly estimated by $\hat f$ a function of $S$ without an integral.
Estimating $\hat f(v, c)$ shares the advantage of fine-scale spatial resolution with \citet{Knox2019-ot}, but in addition enables us to model a particular virus-type $v$, and, as we will see, make use of experimental data in our estimator.


%We fit $g$ using a simple heuristic based.
%Since we wish to weight similar points more highly, we weight distances in cre-space and centroid-space by how well they predict the response variable.
%Due to the fact that the cre-space is defined with respect to the response variable, points are removed from their cre-mean prior to cross-validation.
%Figure shows an example of such an estimated $g$.

%Theoretically, there is a trade off between similarity of the point, and the statistical convergence.
%In the low-sample size setting, it is worth utilizing points from similar groups.


%All of these methods are improvements on the average method.

%is the set of ipsilateral and medialateral brain regions, $R_{IMC}$ is the set of ipsilateral, mediolateral, and contralateral targets,

%We use the experiments to estimate the virus-specific connectivity matrix $\mathcal C_v \in \mathbb R^{|R_{IM}| \times |R_{IMC}|}$.
%$R_{IM}$ is the set of ipsilateral and medialateral brain regions, $R_{IMC}$ is the set of ipsilateral, mediolateral, and contralateral targets, and $v$ is a cre-line. 

%Each row of this matrix represents the fraction of projected virus to a structure $r$.

%Given a region $r$, we estimate the projection vector ${\mathcal C_v}_r$, the row corresponding to the region, both $v$ and the spatial position of $r$.

%This estimate is 
%\begin{equation}
%\hat {C_v}_{r} = \sum_{c \in r} \hat y (v, c) dc.
%\end{equation}
%Here, $\hat y(v, c)$ is the estimate projection pattern of virus-type $v$ at spatial position $c$ within structure $r$.
%In our experiments, this sum is discretized at the voxel level.


%Given a centroid position $c$ and cre-line $v$, we call %our estimator of the projection $\hat y(v, c)$, which is the local weighted average
%\begin{align*}
%\hat y(v, c) &= \sum_{i \in R(c)} \tilde w_\gamma(v,c,c(i),v(i)) y_i
%\end{align*}
%with
%\begin{align*}
%    \tilde w_\gamma(v,c,c(i),v(i)) &= \frac{w_\gamma(v,c,c(i),v(i))}{\sum_{i \in R(c)} w_\gamma(v,c,c(i),v(i))} \\
%\end{align*}
%In this average, nearby experimental projections $y_i$ are averaged with weights determined with respect to the cre-lines $v(i) \in C$ and injection centroid positions $c(i) \in \mathbb R^3$ in the training data.
%In contrast with the estimator in \citet{Knox2019-ot}, we define these weights to depend on cre-line as well as centroid positions.
%In particular,
%\begin{align*}
%    w_\gamma(v,c,c(i),v(i)) &= \exp(- \gamma \hat l ( \Delta c_i,\Delta y_v (i))) % \bar y_{r(c)} (v(i)) - \bar y_{r(c)} (v)) )
%\end{align*}
%where, defining 
%\begin{align*}
%\Delta c_i &= c(i) - c \\
%\Delta y_v (i) &= \bar y_{r(c)} (v(i)) - \bar y_{r(c)} (v) \\
%\delta c_i &= \|\delta c_i\|_2^2 \\
%\delta v_i &= \|\Delta y_v (i)\|_2^2
%\end{align*}
%we have
%\begin{align*}
%&\hat l (\Delta c_i, \Delta y_v (i)) = \sum_{r \in R(c)} \sum_{j,k \in r} \frac{w_{\gamma'}'( \delta c_{jk} ,\delta c_i, \delta v_{jk} , \delta v_i)}{1} \|y_j - y_k\|_2^2 
%= \|y_i - y_j\|
%\| c - c(i) \|_2^2, \|\bar y(v, r(c)) - \bar y(v(i))\|_2^2)
%\end{align*}
%where
%\begin{align*}
%w_{\gamma'}'( \delta c_{jk} ,\delta c_i, \delta v_{jk} , \delta v_i) &= \exp(- \gamma' ( \delta c_{jk} - \delta c_i + \delta v_{jk} - \delta v_i) ).
%\end{align*}
%\hat l ( [c(i) - c, \bar y_{r(c(i))} (v(i)) - \bar y_{r(c(i))} (v(i))], l) \\

%\end{equation}

\newpage


\newpage

\begin{figure}
\begin{adjustbox}{width=.75\columnwidth,center}
\begin{tabular}[t]{cc}
\subfloat[]{
\begin{tikzpicture}[
        execute at begin picture={
          \useasboundingbox (-4,-1) rectangle (4,1);
         }
         ,
  node distance = 8mm and 1mm,
job/.style args = {#1/#2}{
    rectangle, rounded corners, draw=black, very thick,
    minimum height=3em, text width=14em, align=center,
    label={[font=\sffamily,anchor=north]above:\textit{#1}},
    label={[font=\sffamily,anchor=south]below:\textbf{#2}},
    node contents ={}},
    arr/.style = {semithick, -Stealth}
                    ]
\node (n1)  [job=Predict ${\hat y_r \in \mathbb R_{\geq 0}^{|T|}}$/${c, v, s(c)}$];
\node (n21) [job=Voxel /$c$,below left=of n1.south];%$c(i)$/ 2 ,below  left=of n1.south];
\node (n22) [job= Target encode discrete covariates/ ${\mu_{s,v}} $,below right=of n1.south];%$v(i), s(i)$/compute mean $\mu_{s,v}$,below right=of n1.south];
\node (n31) [job=Get centroid distances/${d_{cen} = \{\|c(i') - c\|_2^2 \}} $,below =of n21.south];%$c(i)$/ 2 ,below  left=of n1.south];
\node (n32) [job= Get cre distances/ ${d_{cre} =\{\|\mu_{s,v}(i) - \mu_{s,v}\|_2^2  } $,below =of n22.south];%$v(i), s(i)$/compute mean $\mu_{s,v}$,below right=of n1.south];

\node (n4)  [job=Get weights/${w  = g_M({d_{cen}, d_{cre}})}$,below right=of n31.south];
\node (n5)  [job= Predict/{$\hat y_r = \widehat{NW}(w)$},below=of n4];
%
\coordinate[below=4mm of n1.south] (aux1);
\coordinate[above=4mm of n4.north] (aux2);
%
\draw[arr]  (n1) -- (aux1) -| (n21);
\draw[arr]  (aux1) -| (n22);
\draw[arr]  (n22) -- (n32);
\draw[arr]  (n21) -- (n31);
\draw[arr]  (n31) |- (aux2)
            (n32) |- (aux2) -- (n4);
\draw[arr]  (n4)  -- (n5);
%\caption{2+2}
   \end{tikzpicture}
   }
\hspace{3cm}

   &
   %\subfloat[]{
   %\centering
    \vspace{0cm}
    \subfloat[]{
    \includegraphics[width = 2.1in]{figs/figsforpres/315_summary_scatter.png}
    %\caption{Isocortex loss distribution}
    \label{fig:my_label} 
    }
    \\
    &
    %\begin{figure}
    %\centering
    \subfloat[]{
    \includegraphics[width = 2.1in]{figs/figsforpres/315_summary_surface.png}
    %\caption{$\hat g$ fit to expected-loss using shape-constrained splines}
    %\captionof{figure}{Caption for image}
    \label{fig:my_label}
    %\end{figure}
    }
    \vspace{1cm}
    \end{tabular}
       
 %  \end{subfigure}
 \end{adjustbox}
 \caption{The Expected-Loss estimator}
 \end{figure}
%\end{figure}
%\end{minipage}
%\end{minipage}
%\end{figure}
%\end{tabular}
\newpage


%\subsubsection{The expected-loss estimator}
\paragraph{Justification of shape constraint}

The shape-constrained expected-loss estimator introduced in this paper is, to our knowledge, novel.  It should be considered an alternative method to the classic weighted kernel method. While we do not attempt a detailed theoretical study of this estimator, we do establish the need for the shape constraint in our spline estimator. Though this fact is probably well known, we prove a (slightly stronger) version here for completeness.
%This fact is probably well known, but we prove slightly stronger version of this fact here for completeness.

%The estimated function of distances must be concave, monotonically-increasing, and normalized in order to be a valid distance function on the combined space.  

Given a collection of metric spaces $X_1, ... X_n$ with metrics $d_1 ... d_n$ (e.g. $d_{centroid}, d_cre$), and a function $f: (X_1 \times X_1) ... \times (X_n  \times X_n) = g(d_1(X_1 \times X_1),... d_n(X_n \times X_n))$, then then $f$ is a metric iff $g$ is concave, non-decreasing and $g(d) = 0 \Longleftrightarrow d = 0$.

We first show $g$ satisfying the above properties implies that $f$ is a metric.
\begin{itemize}
    \item The first property of a metric is that $f(x,x') = 0 \Longleftrightarrow x = x'$.  The left implication: $x = x' \implies f(x_1, x_1', ... x_n, x_n') = g(0,....,0)$, since $d$ are metrics.  Then, since $g(0) = 0$, we have that $f(x,x') = 0$. The right implication: $f(x,x') = 0 \implies  d = 0 \implies x = x'$ since $d$ are metrics.
    \item The second property of a metric is that $f(x,x') = f(x',x)$. This follows immediately from the symmetry of the $d_i$, i.e. $f(x,x') = f(x_1, x_1', ... x_n, x_n') = g(d_1(x_1, x_1'), ... d_n(x_n, x_n')) = g(d_1(x_1', x_1), ... d_n(x_n', x_n)) =  f(x_1', x_1, ... x_n', x_n) = f(x',x)$.
    \item The third property of a metric is the triangle inequality: $f(x, x') \leq f(x, x^*) +  f(x^*, x') $.  To show this is satisfied for such a $g$, we first note that $f(x,x') = g(d(x,x')) \leq g(d(x, x^*) + d(x^*, x')) $ since g is non-decreasing and by the triangle inequality of $d$. Then, since $g$ is concave, $g(d(x, x^*) + d(x^*, x')) \leq g(d(x, x^*)) + g(d(x^*, x')) = f(x,x^*) + f(x^*, x')$.
    %or $ f(x_1, x_1', ... x_n, x_n') \leq f(x_1, x_1^*, ... x_n, x_n^*) + f(x_1', x_1^*, ... x_n', x_n^*)$.  To show this is satisfied for such a $g$, we first note $f(x_1, x_1', ... x_n, x_n') = g(d_1(x_1, x_1'),... d_n(x_n ,x_n')) \leq g(d_1(x_1, x_1*) + d_1(x_1', x_1*) ,... , d_n(x_n ,x_n*) + d_n(x_n' ,x_n*))$ since g is non-decreasing and by the triangle inequality on d. Then $g(d_1(x_1, x_1*) + d_1(x_1', x_1*) , ... , d_n(x_n , x_n*) + d_n(x_n' ,x_n*)) < f(x_1, x_1*, ... x_n, x_n*) + f(x_1', x_1*, ... x_n', x_n*)$ since $g$ is concave.
    
    %Lemma: h(a+ b) < h(a) + h(b) for a concave increasing function h.  
%Pf: h(a+b) - h(b) < h(a) - h(0) since rate of increase is decreasing.
\end{itemize}

We then show that $f$ being a metric implies that $g$ satisfies the above properties.
\begin{itemize}
    \item The first property is that $g(d) = 0 \Longleftrightarrow d = 0$. We first show the right implication: $g(d) = 0$, and $g(d) = f(x,x')$, so $x = x'$ (since $f$ is a metric), so $d = 0$. We then show the left implication: $d = 0 \implies x = x'$, since $d$ is a metric, so $f(x,x') = 0,$ since $f$ is a metric, and thus $g(d) = 0$.
    %\item 
    %$f$ is a metric, so $f(x,x') = 0 \Longleftrightarrow x = x'$.  Then, since $f(x,x') = g(d (x,x') )$ which $\implies g(0,\dotsc 0) = 0$ since $g$ is increasing.
    \item The second property is that $g$ is non-decreasing. We proceed by contradiction.
    Suppose g is decreasing in argument $d_1$ in some region $[l, u]$ with $0 < l< u$.
    Then $g(d_1(0, l), 0) \geq g(d_1(0, 0), 0) + g(d_1(0, u), 0) = g(d_1(0, u),0)$, which violates the triangle inequality on f. Thus, decreasing $g$ means that $f$ is not a metric, so $f$ a metric implies non-decreasing $g$.
    \item The final property is that $g$ is concave. We proceed by contradiction. Suppose $g$ is strictly convex. Then there exist vectors $d, d'$ such that $g(d + d')  < g(d) + g(d')$.  Assume that $d$ and $d'$ only are non-zero in the first position, and $d = d(0, x), d' = d(0,x')$.  Then, $f(0,x) + f(0,x') <  f(0,x+ x')$, which violates the triangle inequality on $f$.  Therefore, $g$ must be concave.
\end{itemize}

\subsubsection{Establishing a lower detection limit}
\label{supp:methods_lower}

The lower detection limit of our approach is a complicated consequence of our experimental and analytical protocols.
For example, the Nadaraya-Watson estimator is likely to generate many small false positive connections, since the projection of even a single experiment within the source region to a target will cause a non-zero connectivity in the Nadaraya-Watson weighted average.
On the other hand, the complexities of the experimental protocol itself and the image analysis and alignment can also cause spurious signals.
Therefore, it is of interest to establish a lower-detection threshold below which we have very little power-to-predict, and set estimated connectivities below this threshold to zero.
This should make our estimated connectivities more accurate, especially in the biologically-important sense of sparsity.

We establish this limit with respect to the sum of Type 1 and Type 2 errors
\begin{eqnarray*}
\tau = \sum 1_{f(s,t,c) = 0} 1_{\hat f(s,t,c) > 0} + 1_{f(s,t,c) > 0}  1_{\hat f(s,t,c) = 0} .
\end{eqnarray*}

\subsection{Decomposing the connectivity matrix}

We utilize non-negative matrix factorization (NMF) to analyze the principal signals in our connectivity matrix.
Here, we review this approach as applied to decomposition of the distal elements of the estimated connectivity matrix $\widehat {\mathcal C}$ to identify $q$ connectivity archetypes.
Aside from the NMF program itself, the key elements are selection of the number of archetypes $q$ and stabilization of the tendency of NMF to give random results over different initialization. 

\subsubsection{Non-negative matrix factorization}

Given a matrix $X \in \mathbb R_{\geq 0}^{a \times b}$ and a desired latent space dimension $q$, the non-negative matrix factorization is
\begin{eqnarray*}
NMF(X, q) = \arg \min_{W \in \mathbb R_{\geq 0}^{a \times q} ,H_{\geq 0}^{q \times b }} \| (  X  - WH)\|_2^2.
\end{eqnarray*}
NMF creates a useful decomposition since $X$ is in the positive orthant, and PCA cannot not apply.
There is no orthogonality without sparsity.

We note the existence of NMF with alternative norms for certain marginal distributions, but leave utilization of this approach for future work \citep{Brunet2004-gi}.
%The NMF objective function
%\begin{eqnarray*}
%\arg \min_{W \in \mathbb R_{\geq 0} ,H_{\geq 0} } \| (\hat{  \mathcal C}  - WH)\|_2^2
%\end{eqnarray*}
We can also apply a mask $1_M \in \mathbb R^{S \times T}$ of ones and zeros and solve
\begin{eqnarray*}
\arg \min_{W \in \mathbb R_{\geq 0} ,H_{\geq 0} } \| 1_M \odot  ((\hat{  \mathcal C}  - WH))\|_2^2
\end{eqnarray*}
For us, such a mask serves for two purposes.
First, it enables computation of the NMF objective while excluding self and nearby connections.
These connections are both strong and linearly independent, and so would dominate the $NMF$ reconstruction error.
Long range connections are more biologically interesting or cell-type dependent.
Second, it enables cross-validation based selection of the number of retained components.
%$1_{\text{dist}\geq 1500 \mu m}$

\subsubsection{Cross-validating NMF}

Perhaps surprisingly, cross-validation techniques may also be applied to unsupervised learning problems.
These techniques are somewhat standard, but not entirely well-known, so we review them here, in particular as they apply to the NMF problem.
A NMF model is first fit on a reduced data set, and an evaluation set is held out.
After random masking of the evaluation set, the loss of the learned model is then evaluated on the basis of successful reconstruction of the held-out values.
This procedure is performed repeatedly, with different held out regions and random mask at different dimensionalities $l$, to determine to point past which additional hidden units provide no reconstructive value.

That is, given a matrix $X \in \mathbb R^{S \times T}$ we can decompose $X \sim d(e(X))$ where $e(X)$ is some map that encodes $X$ in a learned representation, and $d$ is the decoding reconstruction map.
In our case, $d$ is simply left multiplication by $W$, and $e$ is the solution of a regularized non-negative least squares optimization problem
\begin{align*}
H := e_W(X) = \arg \min_{\beta} \|X - W \beta\|_2^2.
\end{align*}
%We note 
The form of this solution particularly motivates our cross-validation estimator.

Recall that in supervised learning, the learned model is $Y \sim f(X)$.
Standard cross-validation removes elements of $X$, fits $f$, and then uses the $f$ learned from part of the data to predict $Y$.
A good $f$ will have low error on the training data, and also low error on the test data, indicating that it has not overfit.
Although there is no assumed dichotomy between $X$ and $Y$ in unsupervised learning, for techniques like autoencoders, the above paradigm still applies, i.e., one can still hold out values of $X$.
We can then estimate 
\begin{align*}
\arg \min_{d,e} \widehat E(l(X, d_{X^C}(e_{X^C}(X)))) = \sum_{r=  1}^R l(X_r, d_{X^{C_r}}(e_{X^{C_r}}(X_r)))
\end{align*}
over $R$ random samples of rows of $X$.
However, in our setting, since computing $e(X)$ on the test rows amounts to fitting a non-negative least squares w.r.t. $W$, so the negative effects of an overfit model can simply be optimized away from.
Thus,  the standard solution is to generate uniformly random masks $1_{M(p)} \in \mathbb R^{S \times T}$ where
\begin{align*}
1_{M(p)} (s,t) \sim \text{Bernoulli(p)}.
\end{align*}
%randomly mask $\widehat {\mathcal C}$, we can evaluate the 
Our cross-validation error is then
\begin{align*}
\epsilon_q &= \frac{1}{R} \sum_{r = 1}^R (\|1_{M(p)_r^C} \odot X - \widehat d_q(\widehat e_q (1_{M(p)_r^C} \odot X ))\|_2^2 
\end{align*}
where
\begin{align*}
\widehat d_q, \widehat e_q &= \widehat{\text{ NMF}}(1_{M(p)_r} \odot X, q)).
\end{align*}
The optimum number of components is then
\begin{align*}
    \widehat q = \arg \min_q \epsilon_q.
\end{align*}

\subsubsection{Stabilizing NMF}

The NMF program is non-convex, and, empirically, individual replicates will not converge to the same optima.
One solution therefore is to run multiple replicates of the NMF algorithm, cluster the resulting vectors.
This approach raises the questions of how many clusters to use, and how to deal with stochasticity in the clustering algorithm itself.
We address this issue through the notion of clustering stability \citep{Von_Luxburg2010-lu}.

The clustering stability approach is to generate $L$ replicas of k-cluster partitions $\{C_{kl} : l \in 1 \dots L\}$ and then compute the average dissimilarity between clusterings
\begin{align*}
\xi_k &= \frac{2}{L(L - 1)} \sum_{l = 1}^{L} \sum_{l'= 1}^{l}  d(C_{kl}, C_{kl'}).
\end{align*}
Then, the optimum number of clusters is 
\begin{align*}
\hat k &= \arg \min_k \xi_k.
\end{align*}
A review of this approach is found in \citet{Von_Luxburg2010-qe}.
Intuitively, archetype vectors that cluster together frequently over clustering replicates indicate the presence of a stable clustering.
For $d$, we utilize the adjusted Rand Index - a simple dissimilarity measure between clusterings.
Note that we expect to select slightly more than the $q$ components suggested by cross-validation, since archetype vectors which appear in one NMF replicate generally should appear in others.
We then select the $q$ clusters with the most archetype vectors - the most stable NMF results - and take the median of each cluster to create a sparse representative archetype.

%http://alexhwilliams.info/itsneuronalblog/2018/02/26/crossval/






%\comment{ 
%The limit-of-detection problem is common in a variety of scientific fields, but statistical methodology is not 
%We utilize the zero-inflated adjusted Kendall-$\tau$ statistic

%\begin{align*}
%    \tau_0 = p_{11}^2 \tau_11 + 2(p_{00} p_{11} - p_{10} p_{01}),
%\end{align*}
%where \citep{Pimentel2009KendallsTA, Albasi2018-iv}
%}

\newpage

\section{Supplemental Experiments}

\subsection{Establishing a lower limit of detection}
\label{supp:exp_lower}

\begin{figure}[H]
    \centering
    \includegraphics[width = 7in]{../figures/Threshold}
    \label{fig:threshold}
    \caption{$\tau$ at different limits of detection.}
\end{figure}


\subsection{Loss subsets}

The 
\subsection{Matrix Factorization}

\begin{figure}[H]
    \centering
    \includegraphics[width = 7in]{figs/test_train_0603.png} 
    \label{fig:distances}
    \caption{Train and test error across $2$ \skcomment{increase} replicates using NMF decomposition.}
\end{figure}

\begin{figure}[H]
    \centering
    \includegraphics[width = 7in]{figs/nmfcluster_0610.png} 
    \label{fig:distances}
    \caption{Stability of NMF results across replicates.}
\end{figure}
%\section{Cross-validation for unsupervised learning}












%\section{Data}\label{sec:supp_structure_cre}

%\begin{figure}
%    \centering
%    \includegraphics[width = 7in]{figs/CB centroid densityoct12.png}
%    \caption{Caption}
%    \label{fig:my_label}
%\end{figure}



%\begin{table}[]
%    \centering
%    \begin{tabular}{c|c}
%         &  \\
%         & 
%    \end{tabular}
%    \caption{Caption}
%    \label{tab:cross-validation}
%\end{table}

%However, for alternate models, such as the major-structure divided model from \citet{Knox2019-ot}, the potential evaluation set is larger. In order to compare between methods, we therefore restrict to the smallest set of evaluation indices, which is to say, virus-leaf combinations that are present at least twice.  This means that in some cases, our training set exceeds our evaluation set in size. 


\newpage
\section{Competing Interests}
\label{sec:comp}
This is an optional section. If you declared a conflict of interest when you submitted your manuscript, please  use this space to provide details about this conflict.

\newpage
\bibliography{connectivity}

\newpage
\section{Technical Terms}

\textbf{Technical Term} a key term that is mentioned in an NETN article and whose usage and definition may not be familiar across the broad readership of the journal. 

\textbf{Cre-line}  Refers to the combination of cre-recombinase expression in transgenic mouse and cre-induced promotion in the vector that induces labelling of cell-class specific projection. 

\textbf{Cell class} The projecting neurons targeted by a particular cre-line 

\textbf{Structural connectivities}  connectivity between structures 

\textbf{Voxel} A $100 \mu m$ cube of brain. 

\textbf{Structural connection tensor}  Connectivities between structures given a neuron class

\textbf{dictionary-learning} A family of algorithms for finding low-dimensional data representations.

\textbf{Shape constrained estimator} A statistical estimator that fits a function of a particular shape (e.g. monotonic increasing, convex).

\textbf{Nadaraya-Watson} A simple smoothing estimator.

\textbf{Connectivity archetypes} Typical connectivity patterns

 \textbf{Expected loss} Our new estimator that weights different features by their estimated predictive power.
 
%All NETN article types require Technical Terms.

%Identify approximately 10 key terms that are mentioned in your article and whose usage and definition may not be familiar across the broad readership of the journal. 
%Provide brief (20-word or less) definitions for each term, avoiding in these definitions the use of jargon, or highly technical or specialized language. 
%When the article is typeset, the Technical Terms will appear in the margins at or near their first mention in the text.

%In your manuscript, bold the first occurrence of each \textbf{Technical Term} and then provide a list of the terms and their definitions at the end of the manuscript after the references. 


\end{document}


%\newpage
