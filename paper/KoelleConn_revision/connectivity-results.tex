\section{Results}
\label{sec:results}

We provide several types of results.
First, we show that the novel expected-loss (EL) estimator performs best in our validation assays.
Second, qualitative exploratory analysis confirms that the Cre-specific connectivity matrices generated using this model are consistent with known biology. 
Third, statistical decomposition of the wild-type connectivity matrix using unsupervised learning shows how archetypal components can combine to produce observed signals.

\subsection{Model evaluation}
\label{sec:model_eval}

Our EL model generally performs better than the other estimators that we consider.
Table \ref{tab:crossvalidation} contains weighted losses from leave-one-out cross-validation of candidate models, such as the NW Major-WT model from \citet{Knox2019-ot}.
The EL model combines the good performance of class-specific models like NW Leaf-Cre in regions like Isocortex with the good performance of class-agnostic models in regions like Thalamus.
Additional information on model evaluation, including class and structure specific performance, is given in Appendix \ref{supp_sec:model-evaluation}.
In particular, Supplementary Table \ref{tab:eval_size} contains the sizes of these evaluation sets in each major structure, and Supplementary Section \ref{supp_sec:loss_subsets} contains the structure- and class specific losses.

\begin{table}[H]
\small
\begin{tabular}{lrrrrrrr}
\toprule
& Mean Leaf-Cre & NW Major-Cre& NW Leaf-Cre & NW Leaf &NW Major-WT  & NW Major & EL \\
$\widehat f$ &           Mean & NW & NW & NW & NW & NW &    EL \\
$\mathcal D$ & $I_c \cap I_L$ & $I_c \cap I_M$ & $I_c \cap I_L$ &  $I_L$ & $I_{wt} \cap I_M$ &  $I_M$ &  $I_L$ \\
\midrule
Isocortex &          0.239 &          0.252 &          0.234 &  0.279 &             0.274 &  0.274 & \textbf {0.228} \\
OLF       &          0.193 &          0.233 &          0.191 &  \textbf{0.135} &             0.179 &  0.179 &  0.138 \\
HPF       &          0.175 &          0.332 &          0.170 &  0.205 &             0.228 &  0.228 & \textbf {0.153} \\
CTXsp     &       \textbf   {0.621} &          \textbf{0.621} &         \textbf {0.621} & \textbf {0.621} &            \textbf {0.621} &  \textbf {0.621} &  \textbf{0.621} \\
STR       &          0.131 &         \textbf {0.121} &          0.128 &  0.169 &             0.232 &  0.232 &  0.124 \\
PAL       &          0.203 &          0.205 &          0.203 &  0.295 &             0.291 &  0.291 &  \textbf{0.188} \\
TH        &          0.673 &          0.664 &          0.673 & \textbf {0.358} &             0.379 &  0.379 &  0.369 \\
HY        &          0.360 &          0.382 &          0.353 &  0.337 &             0.317 &  0.317 &  \textbf{0.311} \\
MB        &          0.168 &          0.191 &          0.160 &  0.199 &             0.202 &  0.202 & \textbf {0.159} \\
P         &          0.292 &          0.292 &          0.292 &  0.299 &             0.299 &  0.299 &  \textbf{0.287} \\
MY        &          0.268 &          0.347 &          0.268 &  0.190 &            \textbf{ 0.189 }& \textbf {0.189} &  0.204 \\
CB        &      \textbf {   0.062} &        \textbf { 0.062 }&         \textbf {0.062 }&  0.068 &             0.112 &  0.112 &  0.068 \\
\bottomrule
\end{tabular}
\caption{Losses from leave-one-out cross-validation of candidate models. \textbf{Bold} numbers are best for their major structure.}
\label{tab:crossvalidation}
\end{table}

\newpage

\subsection{Connectivities}

Our main result is the estimation of matrices $\hat {\mathcal C}_v \in \mathbb R_{\geq 0}^{S \times T}$ representing connections of source structures to target structures for particular cre-lines $v$. 
We confirm the detection of several well-established connectivities within our tensor, although it is our expectation that additional interesting biological processes are also manifest.
The connectivity tensor and code to reproduce it are available at \url{https://github.com/AllenInstitute/mouse_connectivity_models/tree/2020}.
%Note that many entries of these matrices are missing due to lack of experiments.

\subsubsection{Overall connectivity}

Several expected biological projection patterns are evident in the wild-type connectivity matrix $\mathcal C_{wt}$ from leaf sources to leaf targets shown in Figure \ref{fig:full_wt}.
Intraareal connectivities are clear, as are ipsilateral connections between cortex and thalumus.
The clear intrastructural and intraareal connectivities mirror previous estimates in \citet{Oh2014-kh} and \citet{Knox2019-ot} and descriptive depictions of individual experiments in \citet{Harris2019-mr}.

Our estimated wild-type connectivities appear more variable than those in \citet{Knox2019-ot}, which used the NW Major-WT model whose accuracy is evaluated in Table \ref{tab:crossvalidation}.
This is plausibly because of both the layer-specific targeting of the different cre-lines, and also the layer-specificity of the selected model.
Although layer-specificity is a major advantage of including distinct cre-lines, for comparison, we also plot coarser projections between summary-structure sources and targets in the cortex in Figure \ref{fig:cortex_wt}.
These are averages over component layers weighted by layer size.
Grossly congruent with the previous work, these results also exhibit a larger range of connectivities than those in \citet{Knox2019-ot}.
Importantly, as shown in Table \ref{tab:crossvalidation} this finer spatial resolution corresponds to the increased accuracy of our EL model over the NW Major-WT model.

\newpage

\begin{figure}[H]
\centering
    \subfloat[] {
    \label{fig:full_wt}
    \includegraphics[width = \textwidth]{figs/conn_leaf2.png}
    } 
        \newline
       \subfloat[] {
    \label{fig:cortex_wt}
    \includegraphics[width = .5\textwidth]{figs/conn_sum_cortex.png}
    } 
   \caption{Wild-type connectivities.
   \ref{fig:full_wt} Log wild-type connectivity matrix $\log \mathcal {C} (s,t,v_{wt})$.
   \ref{fig:cortex_wt} Log wild-type intracortical connectivity matrix at the summary structure level.}
   \label{fig:connectome}
\end{figure}

\newpage
\subsubsection{Class-specific connectivities}

Source and cell-type combinations which project similarly indicate the network structure underpinning cognition.
Although there is a rich anatomical literature using anterograde tracing data to describe projection patterns from subcortical sources to a small set of targets of interest, most accessible whole brain projection data are from the Allen Mouse Connectivity Atlas (MCA) project used here to generate the connectome models.
Thus, to validate our results while avoiding a circular validation of the data used to generate the model weights, we confirm that these class-specific connectivities exhibit certain known behaviors.
In particular, we focus on several cell types and source areas with extensive previous anatomical descriptions of projections using both bulk tracer methods with cell type specificity and single cell reconstructions: 1) thalamic-projecting neurons in the visual and motor cortical regions, 2) cholinergic neurons in the medial septum and nucleus of the diagonal band (MS/NDB); 3) cholinergic neurons in the caudoputamen, and 4) serotonergic neurons of the dorsal raphe nucleus (DR).
We find that our inferred connections are in agreement with literature on these cell types.

\paragraph{Dependence of thalamic connection on cortical layer.}

Visual cortical areas VISp and VISl and cortical motor areas MOp and MOs are ideal testbeds for our connectivities because there are well-established layer-specific projection patterns that can be labeled with the layer-specific cre-lines from the Allen datasets and others \citet{Jeong2016-dc, Harris2019-mr}, and are also well-represented in our dataset.
Figure \ref{fig:ct_spc} shows that in VISp, the Ntsr1-Cre line strongly targets the core part of the thalamic LGd nucleus while in VISl has a very strong projection to the LP nucleus. In VISp, the the Rbp-4Cre strongly targets LP as well.  
Recall that we display connectivity estimates for structures with at least one injection centroid in the structure.
Thus, the position of non-zero rows in Figure \ref{fig:ct_spc} shows the localization of Rbp4-Cre and Ntsr1-Cre injection centroids to layers 5 and 6 respectively.
This is further examined in Supplemental Section \ref{sec:data}).
Thus, as a heuristic alternative model, to also synthesize information about leafs targeted by different cre-lines, we also display an average connectivity matrix over all cre-lines.
This combined output is not evaluated in our testing, and is only a general stand-in for overall behavior, but provides a useful summary of results.

\paragraph{MS and NDB projections in the Chat-IRES-Cre-neo model.}
Cholinergic neurons in the MS and NDB are well-known to strongly innervate the hippocampus, olfactory bulb, piriform cortex, entorhinal cortex, and lateral hypothalamus \citep{Zaborszky2015-fm, Watson2012TheMN}.
In the Allen MCA, cholinergic neurons were labeled by injections into Chat-IRES-Cre-neo mice.
We first checked the estimated connectome weights to targets in these major brain divisions from MS and NDB.
We observed that all these expected divisions were represented above the 90th percentile of weights.
Recently, a single cell whole brain mapping project using Chat-Cre mice fully reconstructed n=50 cells, revealing these same major targets and also naming additional targets from MS/NDB \citep{Li2018-nu}. We compared our Chat-IRES-Cre connectome model data for MS and NDB with the targets identified by  \citet{Li2018-nu}.
We identified ~150 targets at the fine leaf structure level in the top 10\% of estimated weights. To directly compare our data across studies, we had to merge structures as needed to get to the same ontology level and remove ipsilateral and contralateral information.
After formatting our data, we found 51 targets in the top 10\%; Li et al. reported 47 targets across the 50 cells.
There was good consistency overall between the target sets; 35 targets were shared, 12 were unique to the single cell dataset, and 16 unique to our model data.
We checked whether targets missing from our dataset were because of the threshold level.
Indeed, lowering the threshold to the 75 th percentile confirmed 6 more targets-in-common, and all but 2 targets from \citet{ Li2018-nu} were above the 50th percentile weights in our model.
Of note, the absence of a target in the single cell dataset that was identified in our model data is most likely due to the sparse sampling of all possible projections from only n=50 MS/NDB cells.

\paragraph{CP projections in the Chat-IRES-Cre-neo model.}
Most cells in the caudoputamen (CP) are GABAergic spiny projection neurons.
These cells are also the only type that send projections outside the CP.
Cholinergic interneurons make up 1-2\% of all CP cells and their axon terminals do not extend beyond the CP borders.
We confirmed that the model predictions for connection weights from CP cholinergic cells were consistent with this known anatomy; the connection weight to CP was $\sim2$-fold higher than any other in the top $5\%$.

\paragraph{DR projections in the Slc6a4-Cre\_ET33 model.}

Serotonergic projections from cells in the dorsal raphe (DR) are widely distributed and innervate many forebrain structures including isocortex and amygdala.
In the Allen MCA, serotonergic neurons were labeled using Slc6a4-Cre\_ET33 and Slc6a4-CreERT2\_EZ13 mice.
This small nucleus appears to contain a complex mix of molecularly distinct serotonergic neuron subtypes with some hints of subtype-specific projection patterns \citep{Ren2018-ty, Ren2019-jg,  Huang2019-xh}.
We expect that the Cre lines we used here in the Allen MCA, which utilize the serotonin transporter promoter (Slc6a4-Cre and -CreERT2), will lead to expression of tracer in all the serotonergic subtypes recently described in an unbiased way, but this assumption has not been tested directly.
We compared our model data to a single cell reconstruction dataset consisting of n=50 serotonergic cells with somas in the DR that also had bulk tracer validation \skcomment{INSERT LINK TO FIGURE}.
\citet{ Ren2019-jg} listed 55 targets across the single cell reconstructions.
After processing our data to match the target structure ontology level across studies, we identified 37 targets from the DR with weights above the 90 th percentile; 27 of these targets matched those named by \citet{ Ren2019-jg}.
Overall there was good consistency between targets in olfactory areas, cortical subplate, CP, ACB and amygdala areas, as well in pallidum and midbrain.

The two major brain divisions with the least number of matches are the isocortex and thalamus.
There are a few likely reasons for these observations.
First, in the isocortex, there is known to be significant variation in the density of projections across different locations, with the strongest innervation in lateral and frontal orbital cortices \citet{ Ren2019-jg,}. %e.g., see MCA experiment 480074702 and Figs. 6, 7 in Ren 2019
Indeed, when we lower the threshold and check for weights of the targets outside of the 90\%, we see all but one of these regions (PTLp, parietal cortex which is not frontal or lateral) has a weight assigned in the top half of all targets.
In the thalamus, it was also interesting to observe that our model predicted strong connections to several medial thalamic nuclei (i.e., MD, SMT), that were not targeted by the single cells.
This discrepancy may be at least partially explained by the complex topographical organization of the DR that, like the molecular subtypes, is not yet completely understood.
A previous bulk tracer study that specifically targeted injections to the central, lateral wings, and dorsal subregions of the DR reported semi-quantitative differences in projection patterns \citep{Muzerelle2016-sn}.%( Muzarelle et al. 2016% , Table 3).
Notably, \citet{Muzerelle2016-sn} report that cells in the ventral region of DR project more strongly to medial thalamic nuclei, whereas the lateral and dorsal DR cells innervate more lateral regions (e.g., LGd).
So it is possible that the single cell somas did not adequately sample the entire DR.

\newpage

\begin{figure}[H]
\subfloat[]{
\label{fig:ct_spc}
    \includegraphics[width=.8\textwidth]{figs/visp_mo.png} 
    }
    \newline
 \subfloat[]{
 \label{fig:ct_clust}
    \includegraphics[width=.6\textwidth]{figs/heirarchical.png}
    }
    
    \caption{  Cell-class specificity. \ref{fig:ct_spc} Selected cell-class and layer specific connectivities from two visual and two motor areas.
	 Sources without an injection in the Cre driver line are not estimated due to lack of data for that Cre-line in that structure.
    		\ref{fig:ct_clust}
		Hierarchical clustering of connectivity strengths from visual cortex cell-types to cortical and thalamic targets.
		Cre-line, summary structure, and layer are labelled on the sources.
		Major brain division and layer are labelled on the targets.}
\label{fig:data_ct}
\end{figure}

\newpage

%There are also an increasing number of single neuron brain-wide reconstructions, often with genetic information about the reconstructed cell type, that can serve as ground-truth data for these models as they are made available [BICCN fMOST ref].
%We found only minor differences in the models for the two Slc6a4-Cre lines and focus on results from _ET33 below.

\subsection{Connectivity Analyses}

Cell-class, while often correlated with cortical layer, can be a stronger driver of connectivity than summary structure especially when looking at targets at major brain division level.
Figure \ref{fig:ct_clust} shows a collection of connectivity strengths generated using cre-specific models for wild-type, Cux2, Ntsr1, Rbp4, and Tlx3 cre-lines from VIS areas at leaf level in the cortex to cortical and thalamic nuclei.
We use hierarchical clustering to sort source structure/cell-class combinations by the similarity of their structural projections, and sort target structures by the structures from which they receive projections.
Examining the former, we can see that the layer 6 Ntsr1 Cre-line distinctly projects to thalamic nuclei, regardless of source summary structure.
This contrasts with the tendency of other cell-classes to project intracortically in a manner determined by the source structure.
Similarly, layer 6 targets are not strongly projected to by any of the displayed cell classes.
There are too many targeted summary structures to plot here, but we expect that the source profile of each target clusters by structure.

In this section, we apply non-negative matrix factorization to decompose the long-range wild-type connectivities into linear combinations of archetypal connectivities.
This decomposes the remaining censored connectivity matrix into a linear model based off a relatively small number of distinct signals.
This model is able to capture a large amount of the observed variability, and recovers structure-specific archetypal signals.

These signals are plotted in Figure \ref{fig:nmf_results}, and technical details and intermediate results are given in Supplemental Sections \ref{supp_sec:matrix_factor_methods} and \ref{supp_sec:matrix_factor_results}, respectively.
These details include a cross-validation based method for selecting the number of components, a masking method for focusing only on long range connections, and a stability method for ensuring that the decomposition is reliable across computational replicates.
The plotted decomposition shows that these underlying connectivity archetypes correspond strongly to major brain division in both target and sources.
%However, certain components that predominantly represent connectivity from a given major brain division may also be accessed from other areas.
%For example, the IP and FN regions of CB are strongly associated in \ref{fig:W} with the component projecting to MY in \ref{fig:H}.

Inspection of the reconstructed distal normalized connection strength using the top $15$ components shows qualitatively shows that this relatively sparse decomposition is able to capture much of the observed variability.
%Layer-specific targeting is evident, indicating that the factorization method is detecting cell-type specific signals, even though it is trained only on the wild-type connectivity.
Other connectivity patterns like cortical-cortical and cortical-thalamic are also detected.

\newpage

\begin{figure}[H]
\centering
\hspace{1cm}
\subfloat[]{\includegraphics[width = .85 \textwidth]{figs/H_wt.png}
\label{fig:H}} 
\newline
\begin{tabular}[t]{cc}
\adjustbox{valign=t}{\subfloat[]{
\includegraphics[width = .28\textwidth]{figs/W_wt.png}
\label{fig:W}}
} & 
\adjustbox{valign=t}{\subfloat[]{
\includegraphics[width = .6\textwidth]{figs/conn_leafs_recon.png}
\label{fig:recon}}
} 
\end{tabular}
\caption{Non-negative matrix factorization results $\mathcal C_{wt}^N = WH$ for $q = 15$ components.
\ref{fig:H} Latent space coordinates $H$ of $\mathcal C$.
Target major structure and hemisphere are plotted.
\ref{fig:W} Loading matrix $W$.
Source major structure and layer are plotted.
\ref{fig:recon} Reconstruction of the normalized distal connectivity strength using the top $15$ archetypes.  Areas less than $1500 \mu m$ apart are not modeled, and therefore shown in red.
}
\label{fig:nmf_results}
\end{figure}

\newpage

